%%%%%%%%%%%%%%%%%%%%%%%%%%%%%%%%%%%%%%%%%%%%%%%%%%%%%%%%%%%%%%%%%%%%%%%%
%                                                                      %
% This program is free software; you can redistribute it and/or modify %
% it under the terms of the GNU General Public License as published by %
% the Free Software Foundation; either version 2 of the License, or    %
% (at your option) any later version.                                  %
%                                                                      %
% This program is distributed in the hope that it will be useful,      %
% but WITHOUT ANY WARRANTY; without even the implied warranty of       %
% MERCHANTABILITY or FITNESS FOR A PARTICULAR PURPOSE.  See the        %
% GNU General Public License for more details.                         %
%                                                                      %
% You should have received a copy of the GNU General Public License    %
% along with this program; if not, write to the Free Software          %
% Foundation, Inc., 51 Franklin St, Fifth Floor, Boston,               %
% MA  02110-1301  USA                                                  %
%                                                                      %
%%%%%%%%%%%%%%%%%%%%%%%%%%%%%%%%%%%%%%%%%%%%%%%%%%%%%%%%%%%%%%%%%%%%%%%%
%
%	$Id$
%

\setcounter{remarque-cnt}{1}
\setcounter{example-cnt}{1}
\chapter{\label{reg-exp}Les expressions r{\'e}guli{\`e}res}
\thispagestyle{fancy}

%%%%%%%%%%%%%%%%%%%%%%%%%%%%%%%%%%%%%%%%%%%%%%%%%%%%%%%%%%%%
\section{Introduction - D{\'e}finition}

Une \index{expression r{\'e}guli{\`e}re}expression r{\'e}guli{\`e}re est
une s{\'e}quence de symboles qui r{\'e}pond {\`a} une syntaxe
pr{\'e}cise permettant de d'identifier des cha{\^\i}nes de
caract{\`e}res existantes mais ne permet jamais d'en cr{\'e}er. Dans une
expression r{\'e}guli{\`e}re, on peut trouver tous les caract{\`e}res
sauf le caract{\`e}re de saut de ligne (fin de ligne, \verb=\n=,
{\ASCII}$(10)$, \textsl{line feed}).

Les expressions r{\'e}guli{\`e}res sont utilis{\'e}es dans beaucoup de commandes
{\Unix} ("\index{sed@\texttt{sed}}\texttt{sed}",
"\index{awk@\texttt{awk}}\texttt{awk}", "\index{grep@\texttt{grep}}\texttt{grep}",
etc.). Elles sont aussi {\`a} la base d'outils mutli plateforme pour le \textsc{Web}
comme "\texttt{perl}"\cite{learning-perl,programming-perl,advpgm-perl}.
Leur syntaxe et leur lisibilit{\'e} ne sont pas des plus claires mais elles
constituent, en grande partie, la puissance des Shells.

%%%%%%%%%%%%%%%%%%%%%%%%%%%%%%%%%%%%%%%%%%%%%%%%%%%%%%%%%%%%
\section{Sp{\'e}cification de caract{\`e}res particuliers}

\begin{longtable}{|@{\hspace{1ex}}c@{\hspace{1ex}}|@{\hspace{1ex}}p{10cm}@{\hspace{1ex}}|}
	\hline
	\multicolumn{2}{|r|}{Suite de la page pr{\'e}c{\'e}dente $\cdots$}	\\
	\hline
	Symbole			&	Signification		\\
	\hline
\endhead
	\hline
	Symbole			&	Signification		\\
	\hline
\endfirsthead
	\hline
	\multicolumn{2}{|r|}{Suite page suivante $\cdots$}	\\
	\hline
\endfoot
	\hline
\endlastfoot
	\texttt{x}			&
		un caract{\`e}re pr{\'e}cis (ici le caract{\`e}re "\texttt{x}").		\\
	\texttt{.}			&
		un caract{\`e}re quelconque sauf le caract{\`e}re \textsl{line-feed}.	\\
	\index{\^@\texttt{\^}}\verb=^=		&
		d{\'e}but de ligne.												\\
	\index{\$@\texttt{\$}}\texttt{\$}		&
		fin de ligne.												\\
	\index{\@$\mathtt{\backslash}$}\verb=\=		&
		inhibition de l'interpr{\'e}tation d'un caract{\`e}re sp{\'e}cial.		\\
	\index{[]@\texttt{[]}}\verb=[liste]=	&
		un caract{\`e}re quelconque dans la liste.						\\
	\verb=[^liste]=	&
		un caract{\`e}re quelconque absent de la liste.					\\
\end{longtable}

Le caract{\`e}re "\verb=\=" permet de d{\'e}signer les caract{\`e}res
particuliers, utilis{\'e}s dans la d{\'e}finition d'une expression r{\'e}guli{\`e}re,
qui ne doivent pas {\^e}tre interpr{\'e}t{\'e}s.

L'expression r{\'e}guli{\`e}re "\verb=[liste]=" d{\'e}signe n'importe quel
caract{\`e}re de la liste, mais un et un seul seulement. L'expression
r{\'e}guli{\`e}re "\verb=[^liste]=" d{\'e}signe n'importe quel caract{\`e}re sauf les
caract{\`e}res contenus dans la liste. La liste de ces deux expressions
r{\'e}guli{\`e}res peut avoir deux formats~:
\begin{itemize}
	\item	une suite de caract{\`e}res {\`a} prendre en compte. Par exemple:
			"\verb=[abcABC]=" d{\'e}signe l'un de ces six caract{\`e}res. De m{\^e}me
			"\verb*=[ abcABC]=" d{\'e}signe les m{\^e}me caract{\`e}res que pr{\'e}c{\'e}demment,
			\textbf{l'espace en plus}. Ce type d'expression r{\'e}guli{\`e}re s'appelle
			 \textsl{classe de caract{\`e}res} (\textsl{character class}).
	\item	une s{\'e}quence de caract{\`e}res en ne pr{\'e}cisant que les bornes s{\'e}par{\'e}es par
			le caract{\`e}re "\texttt{-}". Par exemple: <<\verb=[r-v]=" d{\'e}signe
			tous les caract{\`e}res compris entre "\texttt{r}" et "\texttt{v}").
			Ce type d'expression r{\'e}guli{\`e}re s'appelle \textsl{character range}
			(intervalle de caract{\`e}res).
\end{itemize}

%%%%%%%%%%%%%%%%%%%%%%%%%%%%%%%%%%%%%%%%%%%%%%%%%%%%%%%%%%%%
\section{Exemples d'application}

\begin{longtable}{|@{\hspace{1ex}}c@{\hspace{1ex}}|@{\hspace{1ex}}p{10cm}@{\hspace{1ex}}|}
	\hline
	\multicolumn{2}{|r|}{Suite de la page pr{\'e}c{\'e}dente $\cdots$}	\\
	\hline
	Expression r{\'e}guli{\`e}re			&	Signification		\\
	\hline
\endhead
	\hline
	Expression r{\'e}guli{\`e}re			&	Signification		\\
	\hline
\endfirsthead
	\hline
	\multicolumn{2}{|r|}{Suite page suivante $\cdots$}	\\
	\hline
\endfoot
	\hline
\endlastfoot
	\hline
	\texttt{abc}	&	d{\'e}signe la cha{\^\i}ne "\texttt{abc}".	\\
	\hline
	\verb=\\= \verb=\*= \verb=\$=	&
		d{\'e}signe le caract{\`e}re qui suit "\verb=\=" comme la constante caract{\`e}re
		et non comme un caract{\`e}re {\`a} interpr{\'e}ter (ici "\verb=\=", "\verb=*=" et
		"\verb=$=").	\\
	\hline
	\verb=c$=	&
		d{\'e}signe le caract{\`e}re "\texttt{c}" en fin de ligne.	\\
	\hline
	\verb=c\$=	&
		d{\'e}signe la cha{\^\i}ne "\verb=c$=".	\\
	\hline
	\verb=^abc=	&
		d{\'e}signe la cha{\^\i}ne "\texttt{abc}" en d{\'e}but de ligne.	\\
	\hline
	\verb=abc$=	&
		d{\'e}signe la cha{\^\i}ne "\texttt{abc}" en fin de ligne.		\\
	\hline
	\verb=^abc$=	&
		d{\'e}signe la cha{\^\i}ne "\texttt{abc}". C'est la seule chaine de la ligne.\\
	\hline
	\verb=^abc.=	&
		d{\'e}signe la cha{\^\i}ne "\texttt{abc}" en d{\'e}but de ligne suivi de n'importe
		quel caract{\`e}re.	\\
	\hline
	\verb=[Aa]=		&
		d{\'e}signe une cha{\^\i}ne contenant le caract{\`e}re "\texttt{A}" ou "\texttt{a}".	\\
	\hline
	\verb=x[Aa][Bb]=	&
		d{\'e}signe une cha{\^\i}ne contenant le caract{\`e}re "\texttt{x}" suivi de "\texttt{A}"
		ou "\texttt{a}" puis de "\texttt{B}" ou "\texttt{b}".	\\
	\hline
	\verb=[^Aa]=	&
		d{\'e}signe une cha{\^\i}ne ne contenant ni caract{\`e}re "\texttt{A}" ni le
		caract{\`e}re "\texttt{a}".	\\
	\hline
	\verb=[r-v]=	&
		d{\'e}signe tous les caract{\`e}res compris entre "\texttt{r}" et "\texttt{v}"
		("\texttt{r}","\texttt{s}","\texttt{t}", "\texttt{u}", "\texttt{v}").	\\
	\hline
	\verb=[^r-v]=	&
		d{\'e}signe tous les caract{\`e}res sauf "\texttt{r}", "\texttt{s}", "\texttt{t}",
		"\texttt{u}" et "\texttt{v}".	\\
\end{longtable}

%%%%%%%%%%%%%%%%%%%%%%%%%%%%%%%%%%%%%%%%%%%%%%%%%%%%%%%%%%%%
\section{Recherche de caract{\`e}res suivant un nombre d'occurrences}

\begin{definition}{Symboles~:}
\begin{longtable}{|@{\hspace{1ex}}c@{\hspace{1ex}}|@{\hspace{1ex}}p{10cm}@{\hspace{1ex}}|}
	\hline
	\multicolumn{2}{|r|}{Suite de la page pr{\'e}c{\'e}dente $\cdots$}	\\
	\hline
	Symbole		&	Signification		\\
	\hline
\endhead
	\hline
	Symbole		&	Signification		\\
	\hline
\endfirsthead
	\hline
	\multicolumn{2}{|r|}{Suite page suivante $\cdots$}	\\
	\hline
\endfoot
	\hline
\endlastfoot
	\hline
	\index{*@\texttt{*}}\verb=*=	&	recherche du caract{\`e}re point{\'e} de 0 {\`a} $n$ fois.	\\
	\hline
	\index{+@\texttt{+}}\verb=+=	&	recherche du caract{\`e}re point{\'e} de 1 {\`a} $n$ fois.	\\
	\hline
	\index{?@\texttt{?}}\verb=?=	&	recherche du caract{\`e}re point{\'e} 0 ou 1 fois.		\\
	\hline
	\index{|@\texttt{|}}\verb=expr1|expr2=	&
		d{\'e}signe une cha{\^\i}ne correspondant {\`a} l'une ou l'autre des deux expressions
		r{\'e}guli{\`e}res.	\\
	\hline
	\index{()@\texttt{()}}\texttt{(}$\cdots$\texttt{)}	&
		partie d'expression r{\'e}guli{\`e}re. \\
\end{longtable}
\end{definition}

Une fois le caract{\`e}re sp{\'e}cifi{\'e}, on peut s{\'e}lectionner le nombre
d'occurrences d'apparition dans l'expression r{\'e}guli{\`e}re.
\begin{itemize}
	\item	Le symbole "\verb=*=" d{\'e}signe que le caract{\`e}re point{\'e} peut appara{\^\i}tre
			de 0 {\`a} $n$ fois dans la cha{\^\i}ne.
	\item	Le symbole "\verb=+=" d{\'e}signe que le caract{\`e}re point{\'e} doit appara{\^\i}tre
			au moins une fois dans la cha{\^\i}ne.
	\item	Le symbole "\verb=?=" d{\'e}signe que le caract{\`e}re point{\'e} doit appara{\^\i}tre
			au plus une fois dans la cha{\^\i}ne.
	\item	Le symbole "\verb=|=" permet de combiner plusieurs expressions r{\'e}guli{\`e}res.
			La condition est satisfaite si au moins l'une des deux expressions est
			satisfaite.
\end{itemize}

Les parenth{\`e}ses permettent de regrouper des expressions r{\'e}guli{\`e}res, afin de fixer des
priorit{\'e}s.

\begin{example}
\begin{longtable}{|@{\hspace{1ex}}c@{\hspace{1ex}}|@{\hspace{1ex}}p{10cm}@{\hspace{1ex}}|}
	\hline
	\multicolumn{2}{|r|}{Suite de la page pr{\'e}c{\'e}dente $\cdots$}	\\
	\hline
	Expression r{\'e}guli{\`e}re		&	Signification		\\
	\hline
\endhead
	\hline
	Expression r{\'e}guli{\`e}re		&	Signification		\\
	\hline
\endfirsthead
	\hline
	\multicolumn{2}{|r|}{Suite page suivante $\cdots$}	\\
	\hline
\endfoot
	\hline
\endlastfoot
	\hline
	\verb=56*=	&
		d{\'e}signe une cha{\^\i}ne contenant un "\texttt{5}" suivi de 0 ou plusieurs
		"\texttt{6}".	\\
	\hline
	\verb=56+=	&
		d{\'e}signe une cha{\^\i}ne contenant un "\texttt{5}" suivi d'au moins un
		"\texttt{5}".	\\
	\hline
	\verb=56?7=	&
		d{\'e}signe une cha{\^\i}ne contenant un "\texttt{5}" suivi d'un "\texttt{6}" ou pas
		puis d'un "\texttt{7}".	\\
	\hline
	\verb=123|321=	&
		d{\'e}signe une cha{\^\i}ne contenant la s{\'e}quence "\texttt{123}" ou la
		s{\'e}quence "\texttt{321}".	\\
\end{longtable}
\end{example}
