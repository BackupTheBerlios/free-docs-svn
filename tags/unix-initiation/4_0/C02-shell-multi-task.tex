%%%%%%%%%%%%%%%%%%%%%%%%%%%%%%%%%%%%%%%%%%%%%%%%%%%%%%%%%%%%%%%%%%%%%%%%
%                                                                      %
% This program is free software; you can redistribute it and/or modify %
% it under the terms of the GNU General Public License as published by %
% the Free Software Foundation; either version 2 of the License, or    %
% (at your option) any later version.                                  %
%                                                                      %
% This program is distributed in the hope that it will be useful,      %
% but WITHOUT ANY WARRANTY; without even the implied warranty of       %
% MERCHANTABILITY or FITNESS FOR A PARTICULAR PURPOSE.  See the        %
% GNU General Public License for more details.                         %
%                                                                      %
% You should have received a copy of the GNU General Public License    %
% along with this program; if not, write to the Free Software          %
% Foundation, Inc., 51 Franklin St, Fifth Floor, Boston,               %
% MA  02110-1301  USA                                                  %
%                                                                      %
%%%%%%%%%%%%%%%%%%%%%%%%%%%%%%%%%%%%%%%%%%%%%%%%%%%%%%%%%%%%%%%%%%%%%%%%
%
%	$Id$
%

\setcounter{remarque-cnt}{1}
\setcounter{example-cnt}{1}
\chapter{\label{multitask}Le mode multi-t{\^a}che}

%%%%%%%%%%%%%%%%%%%%%%%%%%%%%%%%%%%%%%%%%%%%%%%%%%%%%%%%%%%%%%%%%%%%%
\section{T{\^a}che de fond -- le {\sl background}}

\begin{definition}{Syntaxe}
\begin{tabular}{@{\hspace{1cm}}l}
	{\tt \% ligne-de-commande \&}\\[0.2cm]
	{\tt \% jobs}\\[0.2cm]
\end{tabular}
\end{definition}

Le caract{\`e}re \index{&@\texttt{\&}}"\verb=&=" permet
d'ex{\'e}cuter la commande en arri{\`e}re plan (\textsl{background}) et
permet de lancer pendant ce temps l{\`a} d'autres commandes en avant
plan. Lors de la d{\'e}connexion, tous les processes en arri{\`e}re plan
meurent. En effet, ils ne peuvent qu'exister que si leur p{\`e}re
existe.

Lorsqu'une commande est lanc{\'e}e en arri{\`e}re plan, le shell reporte
{\`a} l'utilisateur le num{\'e}ro de \index{PID@PID (\textsl{Process
IDentifier})}PID ({\sl Process IDentifier}) identifiant le processus
lanc{\'e} en arri{\`e}re plan avant de renvoyer le prompt. Une commande
s'ex{\'e}cutant en arri{\`e}re plan \textbf{ne peut pas} {\^e}tre
arr{\^e}t{\'e}e avec les touches \key{{\sc break}}, \control{C}, etc.
Par contre, une sortie de session (\textsl{logout}) tuera tous les
processus s'ex{\'e}cutant en arri{\`e}re plan.

Il est important qu'un processus lanc{\'e} en arri{\`e}re plan ait ses
entr{\'e}es/sorties redirig{\'e}es explicitement. Il est possible de
voir les processes lanc{\'e}s en arri{\`e}re plan avec la commande
\index{jobs@\texttt{jobs}} "\texttt{jobs}".
 
\begin{remarque}
\control{Z} {\bf ne sert pas {\`a} interrompre un process mais {\`a} le mettre
en attente en arri{\`e}re plan}. Si vous voulez interrompre une commande,
utilisez "\control{C}". Pour plus de renseignements sur les touches
d'interruption, reportez vous {\`a} la section \ref{multi-task-fg-bg}.
\end{remarque}

%%%%%%%%%%%%%%%%%%%%%%%%%%%%%%%%%%%%%%%%%%%%%%%%%%%%%%%%%%%%%%%%%%%%%
\section{\label{multi-task-backq}Substitution de commande - caract{\`e}re {\sl back quote}}

\begin{definition}{Syntaxe}
\begin{tabular}{@{\hspace{1cm}}l}
	{\tt `commande`}\\[0.2cm]
\end{tabular}
\end{definition}

La substitution de commande se fait en utilisant le caract{\`e}re back quote
("\index{`@\texttt{`}}\texttt{`}") pour lancer des commandes {\`a} ins{\'e}rer dans une cha{\^\i}ne de
caract{\`e}res.

La substitution de commandes dans une cha{\^\i}ne de caract{\`e}res est une autre
facilit{\'e} offerte par le shell. Elle permet de capturer la sortie d'une
commande et de l'assigner {\`a} une variable ou de l'utiliser comme un
argument d'une autre commande. Comme beaucoup de commandes {\Unix}
g{\'e}n{\`e}rent une sortie, la substitution de commandes peut {\^e}tre tr{\`e}s
int{\'e}ressante.

A propos de l'efficacit{\'e}, il est plus rapide, dans certains cas,
d'effectuer une substitution de commande plutot que de passer par une
redirection.

\begin{example}
\begin{verbatim}
% echo "my current shell is `basename $SHELL`"
% current_dir=`pwd`
% cd /etc
% cd $current_dir
\end{verbatim}
\end{example}

%%%%%%%%%%%%%%%%%%%%%%%%%%%%%%%%%%%%%%%%%%%%%%%%%%%%%%%%%%%%%%%%%%%%%
\section{Commandes associ{\'e}es}

%%%%%%%%%%%%
\subsection{\label{multi-task-kill}Commande "{\tt kill}"}

\begin{definition}{Syntaxe}
\begin{tabular}{@{\hspace{1cm}}l}
	{\tt kill [-sig\_no] PID [PID ...]}\\[0.2cm]
\end{tabular}
\end{definition}

La commande "\index{kill@\texttt{kill}}\texttt{kill}" peut {\^e}tre
utilis{\'e}e pour tuer n'importe quelle commande, y compris des
commandes lanc{\'e}es en arri{\`e}re plan. "{\tt kill}", plus
pr{\'e}cis{\'e}ment, envoie un signal aux processus d{\'e}sign{\'e}s.
L'action effectu{\'e}e par d{\'e}faut pour un process est de mourir sur
r{\'e}ception de certains signaux. Pour transmettre un signal {\`a} un
processus il faut en {\^e}tre le propri{\'e}taire; "{\tt kill}" ne
peut {\^e}tre utilis{\'e} pour tuer le process d'un autre utilisateur
{\`a} moins d'{\^e}tre le {\sl super user} (utilisateur "{\tt
root}").

Par d{\'e}faut "{\tt kill}" transmet le signal num{\'e}ro 15 au process sp{\'e}cifi{\'e}.
Dans le monde {\Unix}, il n'est pas possible de v{\'e}ritablement tuer un process
comme on pourrait le faire avec {\WindowsNT} et {\OpenVMS}. {\Unix}, en fait,
envoie une requ{\^e}te au processus pour qu'il se termine de lui-m{\^e}me. Il est
possible d'immuniser certaines commandes contre les signaux de terminaison. La
m{\'e}thode la plus efficace pour tuer un process est l'utilisation du signal 9.

Pour avoir la liste des signaux et leur action, reportez-vous {\`a} "{\tt signal(2)}".

\begin{example}
\begin{verbatim}
% cat /usr/man/man1/* >big.file &
[[1] 1234
% kill 1234
[1] + Terminated	cat /usr/man/man1/* >big.file
\end{verbatim}
\end{example}

%%%%%%%%%%%%
\subsection{Commande "{\tt wait}"}

\begin{definition}{Syntaxe}
\begin{tabular}{@{\hspace{1cm}}l}
	{\tt wait [PID]}\\[0.2cm]
\end{tabular}
\end{definition}

La commande "\index{wait@\texttt{wait}}\texttt{wait}" permet de
suspendre l'ex{\'e}cution d'un shell script jusqu'{\`a} ce que le
processus, dont le \index{PID@PID (\textsl{Process IDentifier})}\textsl{PID}
est sp{\'e}cifi{\'e} en argument, se termine. Si aucun \textsl{PID} n'est
sp{\'e}cifi{\'e}, on attendra, dans ce cas, que tous les processus lanc{\'e}s
en arri{\`e}re plan soient termin{\'e}s.

\begin{example}
\begin{verbatim}
% cat /usr/man1/* >big.file &
% find / -print >big.file2 &
% wait
% echo "cat et find sont termin{\'e}s ..."
\end{verbatim}
\end{example}

%%%%%%%%%%%%
\subsection{\label{multi-task-fg-bg}Les commandes "{\tt fg}" et "{\tt bg}"}

\begin{definition}{Syntaxe}
\begin{tabular}{@{\hspace{1cm}}l@{\hspace{1cm}}l}
	{\tt bg PID}				&	{\tt bg} = back ground (arri{\`e}re plan)	\\
	{\tt bg \% num{\'e}ro}			&	\\[0.2cm]
	{\tt fg PID}				&	{\tt fg} = foreground (avant plan)		\\
	{\tt fg \% num{\'e}ro}			&	\\[0.2cm]
\end{tabular}
\end{definition}

Nous avons vu qu'avec la touche "\control{Z}", il {\'e}tait possible de
faire passer une commande \textbf{en attente} en arri{\`e}re plan. {\large
L'ex{\'e}cution est suspendue mais le process est toujours actif}. Si vous
d{\'e}sirez qu'elle continue {\`a} s'ex{\'e}cuter, toujours en arri{\`e}re plan,
utilisez la commande "\index{bg@\texttt{bg}}{\tt bg}".

Pour refaire passer en avant plan une commande qui s'ex{\'e}cute en arri{\`e}re
plan utiliser la commande "\index{fg@\texttt{fg}}{\tt fg}". "\texttt{fg}"
permet aussi de r{\'e}activ{\'e} une commande suspendue par "\control{Z}".
Le processus associ{\'e} devient alors le processus courant jusqu'{\`a} ce que
cette commande soit achev{\'e}e.

Pour plus de renseignements, reportez vous {\`a} "\texttt{sh(1)}".

\begin{example}
\begin{verbatim}
% find / -print >/dev/null
\end{verbatim}
\control{Z}
\begin{verbatim}
[1] 1234 Suspended
% bg %1
[1] 1234 find / -print >/dev/null &
% ls
\end{verbatim}
$\cdots$\\
\begin{verbatim}
%fg %1
find / -print >/dev/null
\end{verbatim}
\end{example}

\begin{definition}{\'{E}quivalence}
\begin{tabular}{|l|l|}
	\hline
		\multicolumn{1}{|c|}{{\Unix}}	&
		\multicolumn{1}{|c|}{{\OpenVMS}}	\\
	\hline \hline
		\control{Z}	&	{\tt SET PROCESS/SUSPEND}	\\
	\hline
		{\tt fg}		&	{\tt RESUME}				\\
	\hline
		{\tt bg}		&	Pas d'{\'e}quivalence.			\\
	\hline
\end{tabular}
\end{definition}

%%%%%%%%%%%%
\subsection{Commandes "{\tt at}"}

\begin{definition}{Syntaxe}
\begin{tabular}{@{\hspace{0.5cm}}l}
	{\tt at [-c | -s | -k] [-m] [-q queuename] [-f file]} \\
	\hspace{1cm}{\tt date [increment] [command | file]}\\[0.2cm]
	{\tt at -r job\_number ...}\\[0.2cm]
	{\tt at -l [-q queuename] [user ...]}\\[0.2cm]
\end{tabular}
\end{definition}

La commande "\index{at@\texttt{at}}\texttt{at}" permet de diff{\'e}rer
l'ex{\'e}cution de travaux. Elle lit sur son entr{\'e}e standard les commandes
qu'elle doit lanc{\'e} {\`a} la date sp{\'e}cifi{\'e}e.

Au lieu d'envoyer les ordres {\`a} ex{\'e}cuter sur l'entr{\'e}e standard, vous pouvez~:
\begin{itemize}
	\item	sp{\'e}cifier directement la commande {\`a} la suite. Dans ce cas, l'entr{\'e}e standard
			de la commande qui aura {\'e}t{\'e} saisie, devra {\^e}tre prise {\`a} partir d'un fichier
			avec les m{\'e}canismes d{\'e}crits {\`a} la section \ref{basnot-stdin}.
	\item	donner le nom d'un fichier contenant les commandes {\`a} ex{\'e}cuter. Celui-ci n'est pas
			lu {\`a} la place de l'entr{\'e}e standard mais est trait{\'e} comme une proc{\'e}dure de commande.
			Il doit donc {\^e}tre accessible en ex{\'e}cution par le Shell (cf. sections
			\ref{cmds-protect} et \ref{basicnot-exec}).
\end{itemize}

La commande "\index{batch@\texttt{batch}}\texttt{ batch}" ex{\'e}cute les travaux seulement lorsque le
niveau de la charge du syst{\`e}me le permet. La commande "\texttt{at}"
redirige automatiquement sa sortie standard et sa sortie d'erreurs
standard dans la boite aux lettres du courrier {\'e}lectronique. Par d{\'e}faut,
la commande "\texttt{at}" utilise le Bourne Shell comme interpr{\'e}teur de
commandes. Si vous d{\'e}sirez changer l'interpr{\'e}teur de commande par
d{\'e}faut, les options suivantes devront {\^e}tre sp{\'e}cifi{\'e}es~:
\begin{center}
\begin{tabular}{|l|l|}
	\hline
		\multicolumn{1}{|c|}{Option}		&
		\multicolumn{1}{|c|}{Shell}			\\
	\hline \hline
		"{\tt -c}"	&	C Shell			\\
	\hline
		"{\tt -k}"	&	Korn Shell		\\
	\hline
\end{tabular}
\end{center}

Le param{\`e}tre "{\tt date}" a le format suivant~:
\begin{center}
{\tt [[CC]AA]MMJJhhmm[.ss]}
\end{center}
avec~:\\
\begin{tabular}{@{\hspace{0.2cm}}l@{\hspace{0.2cm}}l}
	{\tt CC}	&	les deux chiffres des centaines de l'ann{\'e}e,	\\
	{\tt AA}	&	les deux derniers chiffres de l'ann{\'e}e,		\\
	{\tt MM}	&	le mois (01-12),							\\
	{\tt JJ}	&	le jour (01-31),							\\
	{\tt hh}	&	l'heure (00-23),							\\
	{\tt mm}	&	les minutes (00-59),						\\
	{\tt ss}	&	les secondes (00-59).						\\
\end{tabular}

"{\tt CC}" et "{\tt AA}" sont optionnels, l'ann{\'e}e en cours est prise par d{\'e}faut.

Le param{\`e}tre "{\tt date}" admet aussi les formats suivants~:
\begin{itemize}
	\item	La commande "{\tt at}" interpr{\`e}te les valeurs compos{\'e}es d'un ou deux chiffres
			comme {\'e}tant des heures. Elle interpr{\`e}te les valeurs compos{\'e}es de quatre chiffres
			comme {\'e}tant des heures et des minutes. Vous pouvez {\'e}ventuellement s{\'e}parer les heures
			des minutes par le caract{\`e}re deux points "{\tt :}". Dans ce cas, le format sera
			"{`}{\tt hh:mm}>.
	\item	Vous pouvez indiquer le suffixe "{\tt am}", "{\tt pm}" ou "{\tt zulu}".
			Si vous ne sp{\'e}cifiez ni "{\tt am}" ni "{\tt pm}", la commande "{\tt at}"
			utilise une horloge de 24 heures. Si vous sp{\'e}cifiez "{\tt zulu}", le temps 
			universel\footnote{LE temps universel correspond au fuseau horaire de Greenwich.}
			est utilis{\'e}.
	\item	L'un des mots clefs suivants~:
		\begin{itemize}
			\item[$\star$]	"{\tt noon}" pour {\sl midi},
			\item[$\star$]	"{\tt midnight}" pour {\sl minuit},
			\item[$\star$]	"{\tt now}" pour repr{\'e}senter l'instant pr{\'e}sent,
			\item[$\star$]	"{\tt A}" en abr{\'e}viation de "{\tt am}",
			\item[$\star$]	"{\tt P}" en abr{\'e}viation de "{\tt pm}", 
			\item[$\star$]	"{\tt N}" en abr{\'e}viation de "{\tt noon}",
			\item[$\star$]	"{\tt M}" en abr{\'e}viation de "{\tt midnight}".
		\end{itemize}
\end{itemize}

\begin{remarque}
	Le mot clef "{\tt now}" peut {\^e}tre utilis{\'e} uniquement si vous indiquez le param{\`e}tre
	"{\tt date}" ou "{\tt increment}". Dans le cas contraire, le message 
	"{\tt too late}" s'affiche.
\end{remarque}

Le param{\`e}tre "{\tt date}" permet d'indiquer un nom de mois et un
num{\'e}ro de jour ({\'e}ventuellement un num{\'e}ro d'ann{\'e}e pr{\'e}c{\'e}d{\'e} d'une virgule),
ou un jour de la semaine. La commande "{\tt at}" reconna{\^\i}t deux
jours particuliers~: il s'agit de "{\tt today}" et de "{\tt
tomorrow}". Si l'heure indiqu{\'e}e est post{\'e}rieure {\`a} l'heure d'entr{\'e}e de
la commande, "{\tt today}" est utilis{\'e}e par d{\'e}faut en tant que
param{\`e}tre "{\tt date}". Dans le cas contraire, celui-ci prend la
valeur "{\tt tomorrow}". Si vous indiquez un mois ant{\'e}rieur {\`a} celui
en cours sans pr{\'e}ciser d'ann{\'e}e, l'ann{\'e}e suivante est utilis{\'e}e par
d{\'e}faut.

Le param{\`e}tre facultatif "{\tt increment}" peut prendre l'une des 
valeurs suivantes~:
\begin{itemize}
	\item	le signe plus "{\tt +}" suivi d'un nombre et de l'un des mots suivants~:
			"{\tt minute[s]}", "{\tt hour[s]}", "{\tt day[s]}", "{\tt week[s]}",
			"{\tt month[s]}" ou "{\tt year[s]}" (ou tout {\'e}quivalent en anglais).
	\item	le mot "{\tt next}" suivi de l'un des mots suivants~: "{\tt minute[s]}",
			"{\tt hour[s]}", "{\tt day[s]}", "{\tt week[s]}",
			"{\tt month[s]}" ou "{\tt year[s]}" (ou tout {\'e}quivalent en anglais).
\end{itemize}

\begin{definition}{Options}
\begin{tabular}{@{\hspace{0.2cm}}l@{\hspace{0.2cm}}p{6cm}}
	{\tt -c}					&	Indique que le C Shell sera utilis{\'e} pour ex{\'e}cuter
									le travail.\\[0.2cm]
	{\tt -k}					&	Indique que le Korn Shell sera utilis{\'e} pour ex{\'e}cuter le
									travail.\\[0.2cm]
	{\tt -l}					&	Liste les travaux {\`a} ex{\'e}cuter.\\[0.2cm]
	{\tt -m}					&	Envoie un message {\`a} l'utilisateur apr{\`e}s l'ex{\'e}cution de
									la commande.\\[0.2cm]
	{\tt -q} {\sl  queuename}	&	Sp{\'e}cifie la file d'attente dans laquelle on effectue le
									travail.\\[0.2cm]
	{\tt -s}					&	Indique que le Bourne Shell sera utilis{\'e} pour ex{\'e}cuter
									le travail.\\[0.2cm]
	{\tt -r} {\sl job}			&	Supprime des travaux programm{\'e}s par la commande
									"{\tt at}" (le param{\`e}tre "{\tt job}" correspond
									{\`a} un num{\'e}ro affect{\'e} par la commande).\\
\end{tabular}
\end{definition}

Par d{\'e}faut, l'attribution des files d'attente se fait de la fa\c{c}on suivante~:\\
\begin{tabular}{|c|p{6cm}|}
	\hline
	file d'attente	&	type de travaux \\
	\hline \hline
	{\tt a}		&	travaux soumis avec la commande "{\tt at}".\\
	{\tt b}		&	travaux "{\sl batch}".\\
	{\tt c}		&	travaux soumis par le "{\sl cron}".\\
	{\tt d}		&	travaux "{\sl sync}".\\
	{\tt e}		&	travaux Korn Shell.\\
	{\tt f}		&	travaux C Shell.\\
	\hline
\end{tabular}

\begin{remarque}
Les travaux "{\sl batch}" sous {\Unix} ont un sens diff{\'e}rent de celui
habituellement entendu sous {\OpenVMS}. Ceux-ci ne sont ex{\'e}cut{\'e}s seulement
si le syst{\`e}me dispose de suffisamment de ressources. Avec {\OpenVMS}, il est possible
d'affecter un niveau de priorit{\'e} des processus pris en compte par l'ordonnanceur du
syst{\`e}me ({\sl scheduler}). Il est donc possible d'avoir une file d'attente offrant
un niveau bas de ressource (priorit{\'e} faible), les travaux ne seront donc ex{\'e}cut{\'e}s
seulement si le syst{\`e}me dispose de suffisamment de ressources. Par contre, il est
aussi possible d'avoir une file d'attent avec un niveau {\'e}lev{\'e}. Dans ce cas, les travaux
passeront en priorit{\'e}. Ceci peut {\^e}tre utile en fonction de l'environnement
d'exploitation.
\end{remarque}

Pour plus de renseignements, reportez-vous {\`a} "{\tt at(1)}".

\begin{example}
Pour programmer l'ex{\'e}cution d'une commande {\`a} partir du terminal, entrez une commande 
semblable {\`a} l'un des exemples suivants.\\
Si "{\tt uuclean}" se trouve dans le r{\'e}pertoire courrant (ou dans un r{\'e}pertoire sp{\'e}cifi{\'e} dans la 
variable "{\tt PATH}")~:
\begin{verbatim}
% at 5 pm Friday uuclean
% at now next week uuclean
\end{verbatim}
Si "{\tt uuclean}" se trouve dans le r{\'e}pertoire "\verb=$HOME/bin/uuclean="~:
\begin{verbatim}
% at now + 2 days $HOME/bin/uuclean
% at now + 2 days
$HOME/bin/uuclean
\end{verbatim}
\control{D}
\end{example}

\begin{remarque}
Lorsque vous indiquez, sur la ligne de commande, un nom de commande
comme dernier {\'e}l{\'e}ment, vous devez sp{\'e}cifier le chemin d'acc{\`e}s complet si
celle-ci ne se trouve pas dans le r{\'e}pertoire courrant. En outre, la
commande "{\tt at}" ne prend en compte aucun argument pour les
commandes {\`a} lancer.
\end{remarque}

\begin{example}
Pour ex{\'e}cuter la commande "{\tt uuclean}" le 24 Janvier {\`a} 15 heures, entrez l'une des 
commandes ci-dessous~:
\begin{verbatim}
% echo uuclean | at 3:00 pm January 24
% echo uuclean | at 3pm Jan 24
% echo uuclean | at 1500 jan 24
\end{verbatim}
\end{example}

\begin{definition}{\'{E}quivalence}
\begin{center}
\begin{tabular}{|c|c|}
	\hline
		{\Unix}		&	{\OpenVMS}			\\
	\hline \hline
		{\tt at}		&	{\tt submit}			\\
	\hline
\end{tabular}
\end{center}
\end{definition}

%%%%%%%%%%%%%%%%%%%%%%%%%%%%%%%%%%%%%%%%%%%%%%%%%%%%%%%%%%%%%%%%%%%%%
\section{R{\'e}p{\'e}tition de t{\^a}ches~: {\tt crontab}}

%%%%%%%%%%%%
\subsection{Introduction -- Syntaxe}

\index{crontab}{\sl crontab}�st un utilitaire utilis{\'e} pour ex{\'e}cuter un certain nombre de t{\^a}ches
r{\'e}p{\'e}titives. Chaque utilisateur poss{\`e}de sa propre table de t{\^a}ches. Celle-ci est
enregistr{\'e}e dans un r{\'e}pertoire du syst{\`e}me d'exploitation (<<{\tt /var/spool/crontab}").

Cette fonctionnalit{\'e} du syst{\`e}me est associ{\'e} {\`a}~:
\begin{itemize}
	\item	la commande "{\tt crontab}",
	\item	un fichier de description des t{\^a}ches {\`a} effectuer.
\end{itemize}

La commande {\tt crontab} ob{\'e}it {\`a} la syntaxe explicit{\'e}e ci-apr{\`e}s.

\begin{definition}{Syntaxe}
{\tt crontab [ -u {\sl user} ] {\sl file}}\\
{\tt crontab [ -u {\sl user} ] \{ -l | -r | -e \}}
\end{definition}

Le premier format de la commande permet d'enregistrer les commandes {\`a} effectuer .
Le second format permet de consulter et de maintenir la table de description des
t{\^a}ches.
Les options disponibles sont~:
\begin{description}
	\item[{\tt -u user}]\mbox{}\\
		L'option "{\tt -u {\sl user}}" permet de pr{\'e}ciser pour  quel utilisateur, d{\'e}fini
		sur le syst{\`e}me, les op{\'e}rations sont {\`a} effectuer. Par d{\'e}faut, la commande
		{\tt crontab} concernera votre propre table de description.
		
	\item[{\tt -l}]\mbox{}\\
		L'option "{\tt -l}" affiche le contenu de la table de description de
		l'utilisateur concern{\'e} sur la sortie standard.
		
	\item[{\tt -r}]\mbox{}\\
		L'option "{\tt -r}" r{\'e}initialise la table de description de l'utilisateur
		concern{\'e}~: toutes les t{\^a}ches qui y sont d{\'e}finies seront supprim{\'e}es.
		
	\item[{\tt -e}]\mbox{}\\
		L'option "{\tt -e}" permet d'{\'e}diter la table de description de l'utilisateur
		concern{\'e}. L'{\'e}diteur utilis{\'e} sera celui pr{\'e}cis{\'e} par le contenu de la variable
		d'environnement "{\tt EDITOR}" ou bien la variable d'environnement
		"{\tt VISUAL}". Les modifications prendont effet imm{\'e}diatement apr{\`e}s la sortie
		de l'{\'e}diteur de texte. 
\end{description}

Pour plus de pr{\'e}cisions, reportez-vous {\`a} {\tt crontab(1)}.

%%%%%%%%%%%%
\subsection{Fichier de description de t{\^a}ches}

Le fichier de description des t{\^a}ches va contenir les instructions
n{\'e}cessaires au processus de supervision (processus "{\tt
cron(8)}"). Celui d{\'e}crira les op{\'e}rations de la fa\c{c}on suivante "{\sl
ex{\'e}cute cette commande {\`a} cette heure l{\`a} ce jour-ci}".

Dans ce fichier,
\begin{itemize}
	\item	toute ligne commen\c{c}ant par le caract{\`e}re "\verb=#=" est ignor{\'e}e (ligne
			de commentaires),
	\item	les lignes blanches sont ignor{\'e}es,
	\item	les espaces et les tabulations en d{\'e}but de ligne sont ignor{\'e}s.
\end{itemize}

\begin{remarque}
Il n'est pas possible de mettre des commentaires au milieu d'une ligne. Le signe de
commentaire doit donc {\^e}tre le premier caract{\`e}re de la ligne.
\end{remarque}

Chaque ligne de description, hormis les lignes de commentaires, peuvent , soit
d{\'e}finir des variables d'environnement pour l'ensemble des t{\^a}ches {\`a} effectuer,
soit des commandes {\`a} ex{\'e}cuter aux instants sp{\'e}cifi{\'e}s. La d{\'e}finition d'une
variable d'environnement dans le fichier ob{\'e}it {\`a} la syntaxe suivante~:
\begin{quote}
{\tt {\sl nom} = {\sl valeur}}
\end{quote}
o{\`u} "{\sl nom}" repr{\'e}sente le nom de la variable d'environnement et "{\sl
valeur}" repr{\'e}sente la valeur qui lui est affect{\'e}e. Les espaces de part et
d'autre du signe {\sl {\'e}gual} "{\tt =}" sont optionnels. Si la valeur {\`a}
affecter {\`a} la variable est une chaine de caract{\`e}res, il est pr{\'e}f{\'e}rable de la
mettre entre simple quotes ("{\tt '}"�u double quotes ("\verb="=").

{\bf Pour toutes les commandes qui seront ex{\'e}cut{\'e}es~:
\begin{itemize}
	\item	leur sortie standard et leur sortie d'erreur standard redirig{\'e}es vers
			la boite aux lettres de l'utilisateur,
	\item	leur entr{\'e}e standard devra {\^e}tre explicitement pr{\'e}cis{\'e}e si cela est
			nec{\'e}ssaire.
\end{itemize}
}

Certaines variables d'environnement sont automatiquement positionn{\'e}es par le
gestionnaire de t{\^a}ches {\tt cron(8)}~:\\
\begin{tabular}{|c|p{8cm}|}
	\hline
	\multicolumn{1}{|c|}{Variable}	&	\multicolumn{1}{|c|}{Description} \\
	\hline \hline
	{\tt SHELL}		&	repr{\'e}sente l'interpr{\'e}teur de commande par d{\'e}faut. Cette variable est initialis{\'e}e {\`a}
					"{\tt /bin/sh}", donc le Bourne Shell.\\
	\hline
	{\tt LOGNAME}	&	repr{\'e}sente le nom de connexion de l'utilisateur. Par d{\'e}faut, elle
					correspond {\`a} votre entr{\'e}e dans la base des utilisateurs.\\
	\hline
	{\tt USER}		&	a exactement les m{\^e}mes fonctionnalit{\'e}s que la variable "{\tt LOGNAME}". Cependant,
					certains utilitaires issus des syst{\`e}mes BSD\footnote{comme SunOS 4 et Ultrix.} utilisent
					"{\tt USER}" plut{\^o}t que "{\tt LOGNAME}".\\
	\hline
	{\tt HOME}		&	repr{\'e}sente votre r{\'e}pertoire de connexion. Par d{\'e}faut, il correspond {\`a} celui
					d{\'e}fini dans les caract{\'e}ristiques de l'utilisateur.\\
	\hline
	{\tt MAILTO}	&	En plus des variables {\tt LOGNAME}, {\tt HOME} et {\tt SHELL},
					le processus "{\tt cron(8)}" examine le contenu de la variable
					d'environnement "{\tt MAILTO}" afin de savoir s'il est n{\'e}cessaire
					d'envoyer un message donnant le r{\'e}sultat de l'ex{\'e}cution des commandes.
					Si cette variable est d{\'e}finie et non vide, elle doit contenir l'adresse
					courrier de la personne {\`a} qui envoyer les messages. Si elle est d{\'e}finie
					{\bf mais vide}\footnote{Dans ce cas la variable est initialis{\'e}e avec
					la chaine vide de la fa\c{c}on suvante~: ":MAILTO="":".}, aucun
					message ne sera envoy{\'e}.\\
					&	Au cas o{\`u} aucune option se serait pr{\'e}cis{\'e} gr{\^a}ce {\`a} la variable
					"{\tt MAILTO}", le r{\'e}sultat produit sur la sortie standard
					est envoy{\'e} dans la boite aux lettres de l'utilisateur poss{\'e}dant
					cette liste de t{\^a}ches {\`a} ex{\'e}cuter.\\
	\hline
\end{tabular}

\begin{remarque}
Cette fonctionnanlit{\'e} offerte gr{\^a}ce {\`a} la variable "{\tt MAILTO}" peut {\^e}tre
tr{\`e}s utile si vous d{\'e}cidez d'utiliser un autre serveur de messagerie que
"{\tt /usr/lib/sendmail}". Dans ce cas, "{\tt /bin/mail}" sera employ{\'e}
pour assurer l'acheminement des messages. "{\tt /bin/mail}" n'utilisera
pas la notion d'{\sl aliases} de "{\tt /usr/lib/sendmail}".
\end{remarque}

Les lignes de description de t{\^a}ches se composent de plusieurs champs s{\'e}par{\'e}s
par des espaces ou tabulations. Chaque ligne poss{\`e}de~:
\begin{itemize}
	\item	cinq champs pr{\'e}cisant la date et l'heure {\`a} laquelle la commande
			doit {\^e}tre lanc{\'e}e,
	\item	dans le cas o{\`u} la table de description est celle de l'administrateur,
			le nom de l'utilisateur pour lequel on lance la commande\footnote{Cette
			fonctionnalit{\'e} n'est pas disponible sur tous les {\Unix}. On la trouvera
			par exemple sous {\Linux}.},
	\item	la commande {\`a} lancer.
\end{itemize}
Le service "{\tt cron(8)}"\footnote{"{\tt cron(8)}"
est un processus syst{\`e}me s'ex{\'e}cutant en permanence. Dans le jargon {\Unix}, un
tel processus est appel{\'e} "{\sl d{\'e}mon}" ou "{\sl daemon}".} examine toutes les
minutes, les tables de description. Lorsque la date pr{\'e}cis{\'e}e dans une entr{\'e}e 
d'une table de description de t{\^a}ches, le service "{\tt cron(8)}" cr{\'e}e un sous
processus sous l'identit{\'e} de la personne et lance la commande {\`a} l'int{\'e}rieur
de ce contexte. La sp{\'e}cification de la date et de l'heure se compose
des cinq informations suivantes~:\\[0.5cm]
\begin{tabular}{|l|l|}
	\hline
	\multicolumn{1}{|c|}{Champ}	&	\multicolumn{1}{|c|}{Valeurs autoris{\'e}es}	\\
	\hline
	minute				&	0-59 \\
	heure				&	0-23 \\
	jour du mois		&	1-31 \\
	mois				&	1-12 \\
	jour de la semaine	&	0-7\footnote{0 ou 7 repr{\'e}sente "{\sl dimanche}")}\\
	\hline
\end{tabular}

Si l'un de ces champs contient le caract{\`e}re "\verb=*=", celui-ci ne
doit pas {\^e}tre pris en compte, ou bien l'une des bornes inf{\'e}rieure ou sup{\'e}rieure
doit {\^e}tre consid{\'e}r{\'e}e dans la d{\'e}termination de la date. Par exemple~:\\[0.5cm]
\begin{tabular}{|l|@{\hspace{0.2cm}indique~:~}p{8cm}|}
	\hline
	{\tt 0 1 * * *}	&	"{\sl tous les jours {\`a} 01:00}"\\
	{\tt 0 0 * * 1}	&	"{\sl tous les lundis {\`a} minuit}"\\
	{\tt * * * 2 3}	&	"{\sl tous les mardis du mois de f{\'e}vrier {\`a} minuit}"\\
	\hline
\end{tabular}

Pour chaque champ, il est possible de donner~:
\begin{itemize}
	\item	une s{\'e}quence de nombres s{\'e}par{\'e}s par des virgules,
	\item	des plages de nombre, en sp{\'e}cifiant les bornes inf{\'e}rieurs et sup{\'e}rieures
			s{\'e}par{\'e}es par le caract{\`e}re "{\tt -}",
	\item	de combiner les deux possibilit{\'e}s.
\end{itemize}

\begin{example}
\begin{tabular}{|l|@{\hspace{0.2cm}correspond {\`a} la s{\'e}quence ~:~}p{3cm}|}
	\hline
	{\tt 1,5}		&	"{\sl 1, 5}"\\
	{\tt 8-10}		&	"{\sl 8, 9, 10}"\\
	{\tt 0-5,7,9}	&	"{\sl 0, 1, 2, 3, 4, 5, 7, 9}"\\
	{\tt 0-4,8-12}	&	"{\sl 0, 1, 2, 3, 4, 8, 9, 10, 11, 12}"\\
	\hline
\end{tabular}
\end{example}

\begin{remarque}
Aucun espace ne doit {\^e}tre introduit dans les cinq champs associ{\'e}s {\`a} la date et 
{\`a} l'heure de lancement d'une commande.
\end{remarque}

\begin{remarque}
Certaines extensions du service "{\tt cron(8)}" permettent
de modifier l'incr{\'e}ment des plages de nombres. Une plage de nombres suivie de
"\verb=/<number>=>> indique le pas entre les valeurs {\`a} prendre {\`a} l'int{\'e}rieur
de l'intervalle. Par exemple, "\verb=0-23/2=", dans le champ <<{\sl heure}"
sp{\'e}cifie toutes les deux heures d'une journ{\'e}e (c'est-{\`a}-dire
"{\tt 0,2,4,6,8,10,12,14,16,18,20,22}"). Cette option est aussi autoris{\'e}e
{\`a} la suite du caract{\`e}re "{\tt *}". Par exemple, "\verb=*/2=" au niveau du
champ "{\sl heure}", sera identique {\`a} l'exmple pr{\'e}c{\'e}dent, c'est-{\`a}-dire
qu'il signifiera "{\sl toutes les deux heures}".
\end{remarque}

Le dernier champ, c'est-{\`a}-dire le reste de la ligne, sp{\'e}cifie la commande
{\`a} ex{\'e}cuter. {\bf Par cons{\'e}quent, la sp{\'e}cification doit {\^e}tre inscrite sur
une seule ligne}. La commande sera ex{\'e}cut{\'e}e dans le contexte du Bourne Shell
({\tt /bin/sh}) ou celui pr{\'e}cis{\'e} gr{\^a}ce {\`a} la variable d'environnement "{\tt SHELL}".

\begin{remarque}
Le jour de l'ex{\'e}cution d'une commande peut {\^e}tre pr{\'e}cis{\'e} dans deux champs~�
\begin{itemize}
	\item	le jour du mois ($3^e$ champ),
	\item	le jour de la semaine ($5^e$ champ).
\end{itemize}
Si ces deux champs sont renseign{\'e}s, c'est-{\`a}-dire qu'ils ne contiennent pas le
caract{\`e}re "{\tt *}", la commande sera ex{\'e}cut{\'e}e d{\`e}s que la date courrante correspondra
{\`a} l'une des deux possibilit{\'e}s. Par exemple, "{\tt 30 4 1,15 * 5}" indiquera que la
commande devra {\^e}tre ex{\'e}cut{\'e}e {\`a} 04:30 le $1^{ier}$ et le quinze de chaque mois, mais
aussi tous les vendredi.
\end{remarque}

%%%%%%%%%%%%
\subsection{Exemple de fichier de description}

\begin{example}
\begin{verbatim}
# Utiliser /bin/sh pour executer les commandes et non pas celui
# pr{\'e}ciser dans le fichier /etc/passwd.
SHELL=/bin/sh
# Envoyer un message a "bart" pour rediriger la sortie standard
# toutes les commandes.
MAILTO=bart
#
# Commande a lancer 5 minutes apres minuit chaque jour.
5  0  *  *  *    $HOME/bin/daily.job >> $HOME/tmp/out 2>&1
# Commande a lancer a 14:15 le premier de chaque mois -- la sortie
# standard de la commande arrive dans la boite aux lettres de "bart".
15 14 1  *  *    $HOME/bin/monthly
# Commande a lancer a 22:00 tous les jours de la semaine (hors week-end)
0  22 *  *  1-5  if [ -d /var/adm/acct ]; then chacct; else echo "no acct"; fi
# Autres commandes
23 0-23/2 * * *  echo "execute tous les jours 23 minutes apres 0h, 2h, 4h ..."
5  4  *  *  0    echo "execute a 04:05 chaque dimanche"
\end{verbatim}
\end{example}
