%%%%%%%%%%%%%%%%%%%%%%%%%%%%%%%%%%%%%%%%%%%%%%%%%%%%%%%%%%%%%%%%%%%%%%%%
%                                                                      %
% This program is free software; you can redistribute it and/or modify %
% it under the terms of the GNU General Public License as published by %
% the Free Software Foundation; either version 2 of the License, or    %
% (at your option) any later version.                                  %
%                                                                      %
% This program is distributed in the hope that it will be useful,      %
% but WITHOUT ANY WARRANTY; without even the implied warranty of       %
% MERCHANTABILITY or FITNESS FOR A PARTICULAR PURPOSE.  See the        %
% GNU General Public License for more details.                         %
%                                                                      %
% You should have received a copy of the GNU General Public License    %
% along with this program; if not, write to the Free Software          %
% Foundation, Inc., 51 Franklin St, Fifth Floor, Boston,               %
% MA  02110-1301  USA                                                  %
%                                                                      %
%%%%%%%%%%%%%%%%%%%%%%%%%%%%%%%%%%%%%%%%%%%%%%%%%%%%%%%%%%%%%%%%%%%%%%%%
%
%	$Id$
%

\setcounter{remarque-cnt}{1}
\setcounter{example-cnt}{1}
\chapter{Commandes {\'e}volu{\'e}es pour les scripts}

%%%%%%%%%%%%%%%%%%%%%%%%%%%%%%%%%%%%%%%%%%%%%%%%%%%%%%%%%%%%
\section{\label{advcmds-shift}La commande "\texttt{shift}"}

\begin{definition}{Syntaxe}
\begin{verbatim}
shift [n]
\end{verbatim}
\end{definition}

La commande \index{shift@\texttt{shift}}"\texttt{shift}" permet de g{\'e}rer les arguments pass{\'e}s {\`a} un
script au niveau de la ligne de commandes. Les actions effectu{\'e}es sont~:
\begin{itemize}
	\item	d{\'e}calage de toutes les cha{\^\i}nes de la variable "\texttt{*}"
			de "\texttt{n}" positions vers la gauche,
	\item	d{\'e}cr{\'e}mentation de la variable "\texttt{\#}" de "\texttt{n}".
\end{itemize}
Par d{\'e}faut, "\texttt{n}" est positionn{\'e} {\`a} 1.

La commande "\texttt{shift}" est une commande \textbf{interne du Shell}.

%%%%%%%%%%%%%%%%%%%%%%%%%%%%%%%%%%%%%%%%%%%%%%%%%%%%%%%%%%%%
\section{La commande "\texttt{read}"}

\begin{definition}{Syntaxe}
\begin{verbatim}
read variable [variable...]
\end{verbatim}
\end{definition}

La commande \index{read@\texttt{read}}"\texttt{read}" est utilis{\'e}e pour lire les informations sur
l{'}entr{\'e}e standard. S'il y a plus de variables dans la commande
"\texttt{read}" que de valeurs r{\'e}ellement saisies\footnote{N'oubliez pas
l'espace ou "\spacekey{}" est s{\'e}parateur au niveau de la ligne
de commandes.}, les variables les plus {\`a} droites sont assign{\'e}es {\`a}
"\texttt{NULL}" (cha{\^\i}ne vide).

Si l'utilisateur saisit plus de mots qu'il n'y a de variables, toutes les donn{\'e}es
de droite sont affect{\'e}es {\`a} la derni{\`e}re variable de la liste.
Une fois positionn{\'e}es, celles-ci sont acc{\'e}d{\'e}es comme les autres.

La commande "\texttt{read}" est une commande \textbf{interne du Shell}.

\begin{example}
On suppose qu'un script contient la commande suivante~:
\begin{sloppypar}
\centering \verb=read var1 var2 var3=
\end{sloppypar}

\begin{description}
	\item[\textsl{Premier cas}~:]\mbox{}\\
		L'utilisateur saisit la cha{\^\i}ne suivante~:
		\begin{sloppypar}
		\centering \verb*=chaine1 chaine2 chaine3=
		\end{sloppypar}
		Dans ce cas, les variables contiendront~:\\[1.5ex]
		\begin{tabular}{|c|l|}
			\hline
			Variable	&	Valeur			\\
			\hline \hline
			\texttt{var1}	&	\texttt{chaine1}	\\
			\texttt{var2}	&	\texttt{chaine2}	\\
			\texttt{var3}	&	\texttt{chaine3}	\\
			\hline
		\end{tabular}
	\item[\textsl{Deuxi{\`e}me cas}~:]\mbox{}\\
		L'utilisateur saisit la cha{\^\i}ne suivante~:
		\begin{sloppypar}
		\centering \verb*=chaine1 chaine2=
		\end{sloppypar}
		Dans ce cas, les variables contiendront~:\\[1.5ex]
		\begin{tabular}{|c|l|}
			\hline
			Variable	&	Valeur			\\
			\hline \hline
			\texttt{var1}	&	\texttt{chaine1}	\\
			\texttt{var2}	&	\texttt{chaine2}	\\
			\texttt{var3}	&	\texttt{NULL}		\\
			\hline
		\end{tabular}
	\item[\textsl{Troisi{\`e}me cas}~:]\mbox{}\\
		L'utilisateur saisit la cha{\^\i}ne suivante~:
		\begin{sloppypar}
		\centering \verb*=chaine1 chaine2 chaine3 chaine4 chaine 5=
		\end{sloppypar}
		Dans ce cas, les variables contiendront~:\\[1.5ex]
		\begin{tabular}{|c|l|}
			\hline
			Variable	&	Valeur			\\
			\hline \hline
			\texttt{var1}	&	\texttt{chaine1}	\\
			\texttt{var2}	&	\texttt{chaine2}	\\
			\texttt{var3}	&	\verb*=chaine3 chaine4 chaine 5=	\\
			\hline
		\end{tabular}
\end{description}
\end{example}

%%%%%%%%%%%%%%%%%%%%%%%%%%%%%%%%%%%%%%%%%%%%%%%%%%%%%%%%%%%%
\section{La commande "\texttt{expr}"}

\begin{definition}{Syntaxe}
\begin{verbatim}
expr expression
\end{verbatim}
\end{definition}

Les arguments de la commande \index{expr@\texttt{expr}}"\texttt{expr}" sont consid{\'e}r{\'e}s comme une expression.
La commande "\texttt{expr}" {\'e}value ses arguments et {\'e}crit le r{\'e}sultat sur
la sortie standard. "\texttt{expr}", comme toutes les commandes {\Unix},
doit avoir ses arguments \textbf{s{\'e}par{\'e}s par des espaces}.
La premi{\`e}re utilisation de "\texttt{expr}" concerne les op{\'e}rations arithm{\'e}tiques
simples. Les op{\'e}rateurs \verb=+=, \verb=-=, \verb=*= et \verb=/= correspondent
respectivement {\`a} l'addition, {\`a} la soustraction, {\`a} la multiplication et
{\`a} la division.

La seconde utilisation de la commande "\texttt{expr}" concerne la comparaison et
la gestion des cha{\^\i}nes de caract{\`e}res gr{\^a}ce {\`a} l'op{\'e}rateur "\verb=:=".
Celui-ci permet de compter combien de fois appara{\^\i}t, dans l'expression plac{\'e}e
en premier membre, \index{expression r{\'e}guli{\`e}re}l'expression r{\'e}guli{\`e}re sp{\'e}cifi{\'e}e en second membre.
Pour plus de renseignements, reportez vous au chapitre \ref{reg-exp}.

Il est possible d'effectuer des regroupements gr{\^a}ce {\`a} des parenth{\`e}ses, comme
dans toute op{\'e}ration arithm{\'e}tique.

\begin{definition}{{\large Attention}}
Ne pas oublier que le processus \index{shell!m{\'e}canisme
d'{\'e}valuation}d'{\'e}valuation du shell effectue ses op{\'e}rations
avant de passer le contr{\^o}le {\`a} la commande "\texttt{expr}".
Par cons{\'e}quent, les caract{\`e}res "\verb=*=", "\verb=(=" et
"\verb=)=" risquent d'{\^e}tre {\'e}valu{\'e}s par le shell et non
pas pass{\'e} en argument {\`a} la commande "\texttt{expr}". Il
faudra donc les faire pr{\'e}c{\'e}der du caract{\`e}re "\verb=\="
(cf. sections \ref{basic-metacars} et \ref{listfcts}).
\end{definition}


\begin{example}
Exemples valides~:
\begin{verbatim}
% X=3; Y=5
% Z=`expr $X + 4 `
% echo $Z
7
% Z=`expr \( $Z + $X \) \* $Y`
% echo $Z
50
% X=abcdef
% Z=`expr $X : '.*' `
% echo $Z
6
% Z=`expr \( $X : '.*' \) + $Y`
% echo $Z
11
\end{verbatim}

Exemples non valides~:\\[2ex]
\begin{longtable}{l@{\hspace{2ex}}p{8cm}}
	\verb*,Z=`expr $X+4`,		& \\
	\multicolumn{2}{@{\hspace{4ex}}p{8cm}}{
	Tout d'abord, le shell cherche une variable dont le nom est "\texttt{X+4}".
	En supposant qu'elle n'existe pas, la commande "\texttt{expr}" ne re\c{c}oit
	qu'un seul argument~:l'expression "\texttt{X+4}", ce qui n'a aucun sens.
	}
	\\
	\verb*,Z=`expr $X * $Y`,	&	\\
	\multicolumn{2}{@{\hspace{4ex}}p{10cm}}{
	Les variables "\texttt{X}" et "\texttt{Y}" existent bien et contiennent
	des valeurs num{\'e}riques. Par cons{\'e}quent, le r{\'e}sultat de l'{\'e}valuation de
	"\texttt{\$X}" et "\texttt{\$Y}" est valide. Par contre, "\texttt{*}"
	est un m{\'e}tacaract{\`e}re valide du shell (cf. section \ref{basic-metacars}). Il
	doit donc {\^e}tre remplac{\'e} par les fichiers correspondants, c'est-{\`a}-dire tous
	les fichiers du r{\'e}pertoire courant. L'expression devient alors~:
	}
	\\
	\hspace{4ex}\verb*,Z=`expr 3 fic1 fic2 fic3 fic4 fic5 fic6 4`,	&
	\\
	\multicolumn{2}{@{\hspace{4ex}}p{10cm}}{
	en supposant que les variables "\texttt{X}" et "\texttt{Y}" contiennent
	respectivement les valeurs "\texttt{3} " et "\texttt{4}" et que le
	r{\'e}pertoire courant contient les fichiers "\texttt{fic1}", "\texttt{fic2}",
	"\texttt{fic3}", "\texttt{fic4}", "\texttt{fic5}" et "\texttt{fic6}".
	}
	\\
	\verb*,Z=`expr ( $X \* $Y ) + $Z`,	&	\\
	\multicolumn{2}{@{\hspace{4ex}}p{10cm}}{
	Ici, les caract{\`e}res "\texttt{(}" et "\texttt{)}" sont pris en compte
	par le m{\'e}canisme d'{\'e}valuation du shell~: ils indiquent une expression {\`a} ex{\'e}cuter
	dans un processus s{\'e}par{\'e} (cf. section \ref{listfcts}). Il appara{\^\i}t que, ne serait-ce que pour la syntaxe,
	cette expression est invalide pour le shell. C'est donc le shell qui renverra une erreur
	et non pas la commande "\texttt{expr}".
	}
\end{longtable}
\end{example}
