%%%%%%%%%%%%%%%%%%%%%%%%%%%%%%%%%%%%%%%%%%%%%%%%%%%%%%%%%%%%%%%%%%%%%%%%
%                                                                      %
% This program is free software; you can redistribute it and/or modify %
% it under the terms of the GNU General Public License as published by %
% the Free Software Foundation; either version 2 of the License, or    %
% (at your option) any later version.                                  %
%                                                                      %
% This program is distributed in the hope that it will be useful,      %
% but WITHOUT ANY WARRANTY; without even the implied warranty of       %
% MERCHANTABILITY or FITNESS FOR A PARTICULAR PURPOSE.  See the        %
% GNU General Public License for more details.                         %
%                                                                      %
% You should have received a copy of the GNU General Public License    %
% along with this program; if not, write to the Free Software          %
% Foundation, Inc., 51 Franklin St, Fifth Floor, Boston,               %
% MA  02110-1301  USA                                                  %
%                                                                      %
%%%%%%%%%%%%%%%%%%%%%%%%%%%%%%%%%%%%%%%%%%%%%%%%%%%%%%%%%%%%%%%%%%%%%%%%
%
%	$Id$
%

\setcounter{remarque-cnt}{1}
\setcounter{example-cnt}{1}
\chapter{Introduction}

%%%%%%%%%%%%%%%%%%%%%%%%%%%%%%%%%%%%%%%%%%%%%%%%%%%%%%%%%%%%%%%%%%%%%%%%%%%%%%%%%%%%%
\section{D{\'e}finition}

Un programme \index{shell!script}shell est un fichier r{\'e}gulier
contenant des commandes {\Unix} ou bien des commandes
"intrins{\`e}ques" du shell. Les permissions de ce fichier doivent
au moins {\^e}tre lecture et ex{\'e}cution. Pour ex{\'e}cuter un
programme shell, taper le nom du programme apr{\`e}s le prompt {\`a}
condition que le programme shell se trouve dans un des r{\'e}pertoires
sp{\'e}cifi{\'e}s dans la variable \index{variable!PATH@\texttt{PATH}}
\texttt{PATH}. Nous verrons ce point plus en d{\'e}tail {\`a} la
section{call-shell}.

%%%%%%%%%%%%%%%%%%%%%%%%%%%%%%%%%%%%%%%%%%%%%%%%%%%%%%%%%%%%%%%%%%%%%%%%%%%%%%%%%%%%%
\section{Historique des shells}

Le premier Shell d{\'e}velopp{\'e} pour {\Unix} fut appel{\'e} le
\index{shell!sh@\texttt{sh}}Bourne Shell, nom provenant du responsable
de l'{\'e}quipe de d{\'e}veloppement ({\sl Steve Bourne}). Les personnes
de l'universit{\'e} de Berkeley en Californie\footnote{University of
California Berkeley = UCB.} trouv{\`e}rent des lacunes au Bourne Shell
(d'apr{\`e}s eux, trop de donn{\'e}es devaient {\^e}tre saisies) et
d{\'e}velopp{\`e}rent un nouveau Shell~: le
\index{shell!csh@\texttt{csh}}"{\sl C Shell}" \footnote{C comme
Californie et non pas comme le langage.}. Le "{\sl C Shell}" offrit
de nouvelles caract{\'e}ristiques que n'avait pas le Bourne Shell~:
\begin{itemize}
	\item	Premi{\`e}rement, l'utilisateur pouvait rappeler les commandes et les {\'e}diter. La proc{\'e}dure
			n'est pas tr{\`e}s ais{\'e}e\footnote{Les rappels des commandes avec le "{\sl C Shell}" consiste
			{\`a} rappeler la derni{\`e}re commande commen\c{c}cant par une chaine de caract{\`e}res ou bien la $n^{e}$
			commande saisie. Il suffit alors de faire pr{\'e}c{\'e}dent la chaine ou bien le num{\'e}ro de la
			commande par le caract{\`e}re "{\tt !}".} mais fonctionnait et avait, au moins, le m{\'e}rite d'exister
			contrairement au Bourne Shell.
	\item	En second, le "{\sl C shell}" permettait de d{\'e}finir diff{\'e}rents noms pour une
			commande~: les "{\sl aliases}".
	\item	Troisi{\`e}mement, le "{\sl C shell}" permit la g{\'e}n{\'e}ration automatique 
			de fin de nom de fichier. Il permettait de terminer le nom du fichier en tapant
			sur la touche "{\sl Escape}" du clavier.
	\item	En fin, l'initialisation et la terminaision du contexte d'une session utilisateur
			appelle un ou plusieurs fichiers de configuration, en fonction du mode d'appel.
			Il a donc {\'e}t{\'e} distingu{\'e}~:
			\begin{itemize}
				\item[$\star$] l'initialisation d'une session interactive,
				\item[$\star$] l'initialisation de l'environnement li{\'e} {\`a} une proc{\'e}dure ou
								un {\it sous-shell}\footnote{Un {\it sous-shell} correspond
								{\`a} un sous-processus associ{\'e} {\`a} un interpr{\'e}teur de commande
								(ou {\it shell}), lanc{\'e} {\`a} partir d'une session interactive.},
				\item[$\star$] la fin d'une session interactive.
			\end{itemize}.
			Par cons{\'e}quent, il est possible de param{\`e}trer son environnement plus finement.
\end{itemize}

Il pr{\'e}senta donc beaucoup d'autres avantages par rapport au Bourne Shell mais avait deux 
inconv{\'e}nients majeurs~:
\begin{itemize}
	\item	il n'{\'e}tait pas standard,
	\item	il {\'e}tait incompatible avec les programmes Bourne Shell.
\end{itemize}

Finalement, David {\sc Korn} des laboratoires {\sc Bell} d'AT\&T e{\^u}t
le dernier mot. Il {\'e}crivit le \index{shell!ksh@\texttt{ksh}}"{\sl
Korn Shell}" (appel{\'e} K Shell) avec son {\'e}quipe de
d{\'e}veloppeurs. Celui-ci poss{\`e}de les meilleures
caract{\'e}ristiques du Bourne Shell, inclut celles du C Shell tout en
{\'e}tant 100\% compatible avec le Bourne Shell. Il rajouta aussi un
certain nombre de fonctionnalit{\'e}s pour la programmation de
proc{\'e}dures de commandes. Enfin, pour {\sl concurencer} le "{\sl
Korn Shell}", son {\'e}quivalent fut d{\'e}velopp{\'e} avec les
syntaxes du "{\sl C Shell}"~: le "{\tt tcsh}". Ce dernier est un
sur-ensemble du "{\sl C Shell}" avec un effort accrue sur la
souplesse d'utilisation de l'interpr{\'e}teur de commande (utilisation
plus {\it conviviale}, fonctionnalit{\'e}s pour les proc{\'e}dures de
commandes accrues, etc.).

Toutefois, l'{\'e}quipe du "{\sl Bourne Shell}" n'en resta pas
l{\`a}. Ses fonctionnalit{\'e}s furent accrues avec le
\index{shell!bash@\texttt{bash}}"{\tt bash}" signifiant "{\sl
Bourne another Shell}".

\begin{definition}{En conclusion}
Il existe deux grandes familles pour les syntaxes~:
\begin{itemize}
	\item	celle du "{\sl Bourne Shell}", regroupant
			le Bourne Shell, le Korn Shell et le {\tt bash},
	\item	celle du "{\sl C Shell}", regroupant
			le C Shell et le {\tt tcsh}.
\end{itemize}
Nous ne parlerons dans ce document que du "{\sl Bourne Shell}",
premi{\`e}re {\'e}tape vers la cr{\'e}ation de proc{\'e}dures sous
{\Unix}. La correspondance entre les syntaxes du "{\sl Bourne
Shell}" et du "{\sl C Shell}" seront explicit{\'e}es {\`a}
l'annexe \ref{ann-csh-sh}.
\end{definition}

\begin{remarque}
Le Shell POSIX s'appuie sur le "{\sl Korn Shell}". Donc si vous
voulez rester dans la norme POSIX, exigez le "{\sl Korn Shell}".
\end{remarque}


%%%%%%%%%%%%%%%%%%%%%%%%%%%%%%%%%%%%%%%%%%%%%%%%%%%%%%%%%%%%%%%%%%%%%%%%%%%%%%%%%%%%%
\section{Quelques r{\`e}gles et recommandations pour l'{\'e}criture des shells-scripts}

\subsection{\label{pgm-intro-basic-rules}R{\`e}gles {\`a} observer}

\begin{longtable}{l@{\hspace{0.2cm}}p{10cm}}
	\multicolumn{2}{r}{{\sl Suite page suivante $\cdots$}} \\
\endfoot
	R{\`e}gle 1~:	&
		Commentez abondamment et documentez syst{\'e}matiquement vos scripts avec une en-t{\^e}te comprenant~:
		\begin{itemize}
			\item	le nom de l'auteur,
			\item	la date,
			\item	une pr{\'e}sentation des fonctionnalit{\'e}s.
		\end{itemize}
		\\[0.2cm]
	R{\`e}gle 2~:	&
		V{\'e}rifier les param{\`e}tres d'appel, et indiquer l'usage si l'appel est incorrect.
		\\[0.2cm]
	R{\`e}gle 3~:	&
		Assortir toute sortie anormale d'un message d'erreur et d'un appel {\`a} "{\tt exit}"
		avec un num{\'e}ro d'erreur afin de pouvoir {\^e}tre test{\'e} dans tout script appelant.
		\\[0.2cm]
	R{\`e}gle 4~:	&
		Adopter une approche modulaire avec usage de~:
		\begin{itemize}
			\item	fonctions internes,
			\item	scripts externes,
			\item	appel {\`a} des scripts existants.
		\end{itemize}
		\\[0.2cm]
	R{\`e}gle 5~:	&
		Utilisez les minuscules pour les noms de variables internes {\`a} un script ou
		une fonction (hors environnement) et les majuscules pour les variables empil{\'e}es
		dans ou h{\'e}rit{\'e}es de l'environnement.
		\\[0.2cm]
	R{\`e}gle 6~:	&
		Si des param{\`e}tres d'environnement propres au d{\'e}roulement ou {\`a} l'encha{\^\i}nement 
		de scripts sont n{\'e}cessaires, les terminaisons normales et anormales doivent assurer 
		le d{\'e}pilement de ces param{\`e}tres, afin de ne pas polluer inutilement 
		l'environnement utilisateur au fur et {\`a} mesure des appels.
		\\[0.2cm]
	R{\`e}gle 7~:	&
		Si un script fait r{\'e}f{\'e}rence {\`a} des ressources (fichiers) qu'il ne cr{\'e}e pas lui-m{\^e}me, 
		et/ou s'il atteint un certain volume (sup{\'e}rieur {\`a} une page par exemple), r{\'e}digez les 
		traces (jalonnement de "{\tt stdout}") et des messages d'erreurs ("{\tt stderr}")
		sur un m{\^e}me fichier de {\sl log}, portant le m{\^e}me nom que le script avec l'extension
		<<{\tt .log}".
		\\[0.2cm]
	R{\`e}gle 8~:	&
		Si des messages et/ou des erreurs doivent {\^e}tre redirig{\'e}es sur la console\footnote{La console
		est un terminal sp{\'e}cial du syst{\`e}me charg{\'e} d'afficher tous les messages syst{\`e}mes. En g{\'e}n{\'e}ral,
		ce p{\'e}riph{\'e}que est connect{\'e} physiquement {\`a} la machine sous {\Unix}. Elle peut toutefois {\^e}tre
		d{\'e}port{\'e}e de diff{\'e}rentes fa\c{c}ons.}, alors pr{\'e}fixez les du nom du script qui les g{\'e}n{\`e}rent.
		\\[0.2cm]
	R{\`e}gle 9~:	&
		Tout fichier temporaire doit {\^e}tre cr{\'e}{\'e} dans un r{\'e}pertoire {\bf r{\'e}serv{\'e} et distinct} qui sera 
		identifi{\'e} par une variable d'environnement ("{\tt D\_TEMP}" par exemple). {\bf Ce r{\'e}pertoire 
		sera distinct de "{\tt /tmp}"}. La n{\'e}cessit{\'e} de plusieurs r{\'e}pertoires temporaires sera 
		satisfaite par la cr{\'e}ation de sous-r{\'e}pertoires dans celui-ci. Le "{\sl PID}"\footnote{{\sl
		PID~: Process IDentifier.}} (accessible gr{\^a}ce {\`a} la variable "\verb=$$="\footnote{cf. section
		\ref{variables-special-var}.}) du script sera utilis{\'e} {\bf syst{\'e}matiquement} comme suffixe des noms de fichiers
		temporaires g{\'e}n{\'e}r{\'e}s.
		\\
				&
		Toutes les ressources temporaires (sous-r{\'e}pertoires et fichiers) seront {\'e}vacu{\'e}es 
		de cette zone par destruction (commande "{\tt rm}") ou rang{\'e}es ailleurs (commande
		"{\tt mv}") lors des terminaisons.
		\\[0.2cm]
	R{\`e}gle 10~:	&
		Tout script aura comme premi{\`e}re ligne, la description du Shell {\`a} lancer. On aura 
		donc~:
		\\[0.2cm]
				&
		\begin{tabular}{|c|l|}
			\hline
			\multicolumn{1}{|c|}{Shell}	&
			\multicolumn{1}{|c|}{Premi{\`e}re ligne}	\\
			\hline \hline
			Bourne Shell			&	\verb=#!/bin/sh=	\\
			\hline
			C Shell					&	\verb=#!/bin/csh=	\\
			\hline
			Korn Shell				&	\verb=#!/bin/ksh=	\\
			\hline
			TC Shell				&	\verb=#!/bin/tcsh=	\\
			\hline
			Bourne Another Shell	&	\verb=#!/bin/bash=	\\
			\hline
		\end{tabular}
		\\[0.2cm]
				&
		{\bf La ch{\^a}ine de caract{\`e}re sp{\'e}cifiant le nom de l'ex{\'e}cutable doit {\^e}tre {\'e}crite
		sans espaces et d{\`e}s la premi{\`e}re colonne.}
		\\[0.2cm]
\end{longtable}


%%%%%%%%%%%%%%
\subsection{Recommandations}

Le premier souci du d{\'e}veloppeur de {\sl shell script} doit {\^e}tre la portabilit{\'e}.
En effet, il est dispendieux d'avoir {\`a} r{\'e}adapter des proc{\'e}dures d'exploitation {\`a}
chaque introduction d'une nouvelle machine. Les scripts d{\'e}velopp{\'e}s sur une
architecture d'un constructeur sous le m{\^e}me syst{\`e}me d'exploitation ({\Unix}),
doivent se comporter de fa\c{c}on identique sur {\sl SunOS}, {\sl Solaris}, {\sc
HP--UX}, {\sl Irix}, {\sl Digital {\Unix}}, {\sl AIX}, etc. La normalisation POSIX
est l{\`a} pour le permettre.

Les d{\'e}veloppements ne doivent pas {\^e}tre des op{\'e}rations o{\`u} l'on red{\'e}couvre et reconstruit la roue. 
Un usage syst{\'e}matique des utilitaires ad-hoc (de la "{\sl section 1}") est recommand{\'e}. Quand cela s'av{\`e}re 
n{\'e}cessaire pour les traitements de donn{\'e}es symboliques, on utilisera en priorit{\'e} les utilitaires de 
manipulation de fichiers.

On limitera volontairement l'usage des utilitaires "{\tt sed}" et
"{\tt awk}"  {\`a} des s{\'e}quences simples de traitement, d{\`e}s
l'instant o{\`u} l'on a recours {\`a} des expressions
r{\'e}guli{\`e}res. D{\'e}composez les traitements {\`a} effectuer.
Exploitez la philosophie et les possibilit{\'e}s d'{\Unix}~:
\begin{itemize}
	\item	chaque commande s'ex{\'e}cute dans un sous process,
	\item	{\Unix} dispose d'un ensemble de commande tr{\`e}s vastes, chacune ayant une fonction bien 
			pr{\'e}cise (t{\^a}che {\'e}l{\'e}mentaire).
\end{itemize}

Il est beaucoup plus {\'e}conomique d'utiliser plusieurs commandes qui s'encha{\^\i}nent (avec les 
m{\'e}canismes des {\sl pipes}) pour r{\'e}aliser une t{\^a}che complexe. Vous y gagnerez en simplicit{\'e} de 
raisonnement et de mise au point.

N'oubliez pas que le {\sl C Shell}, le {\sl TC Shell}, le {\sl Bourne Another
Shell} (ou {\tt bash}) et le Korn Shell ex{\'e}cutent, {\`a} chaque lancement, les
fichiers suivants~:

\begin{tabular}{|c|l|}
	\hline
	Shell						&	\multicolumn{1}{|c|}{Fichier}	\\
	\hline \hline
	C Shell						&	\verb=~/.cshrc=					\\
	TC Shell					&	\verb=~/.tcshrc=				\\
	Bourne Another Shell		&	\verb=~/.bashrc=				\\
	Korn Shell					&	\verb=~/.kshrc=					\\
	\hline
\end{tabular}

Par cons{\'e}quent, {\`a} chaque lancement d'un script {\sl C Shell}, {\sl TC Shell}, {\sl Bourne Another
Shell} (ou {\tt bash}) ou Korn Shell, le fichier d'initialisation est 
ex{\'e}cut{\'e} par d{\'e}faut.
