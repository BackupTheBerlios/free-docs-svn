%%%%%%%%%%%%%%%%%%%%%%%%%%%%%%%%%%%%%%%%%%%%%%%%%%%%%%%%%%%%%%%%%%%%%%%%
%                                                                      %
% This program is free software; you can redistribute it and/or modify %
% it under the terms of the GNU General Public License as published by %
% the Free Software Foundation; either version 2 of the License, or    %
% (at your option) any later version.                                  %
%                                                                      %
% This program is distributed in the hope that it will be useful,      %
% but WITHOUT ANY WARRANTY; without even the implied warranty of       %
% MERCHANTABILITY or FITNESS FOR A PARTICULAR PURPOSE.  See the        %
% GNU General Public License for more details.                         %
%                                                                      %
% You should have received a copy of the GNU General Public License    %
% along with this program; if not, write to the Free Software          %
% Foundation, Inc., 51 Franklin St, Fifth Floor, Boston,               %
% MA  02110-1301  USA                                                  %
%                                                                      %
%%%%%%%%%%%%%%%%%%%%%%%%%%%%%%%%%%%%%%%%%%%%%%%%%%%%%%%%%%%%%%%%%%%%%%%%
%
%	$Id$
%

\section{Introduction}

Dans ce chapitre de l'annexe, nous allons d{\'e}crire les deux {\'e}diteurs de texte
les plus commun{\'e}ment utilis{\'e}s sous {\Unix}~:
\begin{itemize}
	\item	\index{vi@\texttt{vi}}"{\tt vi}",
	\item	\index{emacs@\texttt{emacs}}"{\tt emacs}".
\end{itemize}

"{\tt vi}" est l'{\'e}diteur de texte de base sur tout syst{\`e}me
{\Unix}. Il est livr{\'e} syst{\'e}matiquement et utilise toutes les notions de
syntaxes vues avec les utilitaires comme "{\tt sed}" et "{\tt
awk}" (cf. sections \ref{adv-fltrs-sed} et \ref{adv-fltrs-awk}). Il
fonctionnera sur tout type de terminal, m{\^e}me sur un terminal
t{\'e}l{\'e}type\footnote{m{\^e}me chose qu'un terminal {\ASCII} classique mais l'{\'e}cran
est remplac{\'e} par une imprimante}. "{\tt vi}" n'est pas r{\'e}put{\'e} pour
sa convivialit{\'e}. Il demande un certain temps d'adaptation. Une fois que
l'on a r{\'e}ussi {\`a} s'y habituer, son utilisation devient ais{\'e}e et toutes
ses fonctionnalit{\'e}s sont tr{\`e}s rapides d'acc{\`e}s.

"{\tt emacs}" est un {\'e}diteur de texte du domaine public, livr{\'e}
maintenant en standard sur certains {\Unix} (comme "Digital
{\Unix}" \footnote{{\Unix} livr{\'e} sur les machines Digital, comme les
{\sl AlphaStations}, {\sl AlphaServers}, etc.},
"{\Linux}"\footnote{{\Unix} du domaine public pouvant tourner sur
les architectures PC-Intel, PowerMacintosh ou compatibles, machines {\`a}
base de processeurs MIPS et Alpha} "Irix"\footnote{{\Unix} de
Silicon Graphics}). Dans le cas contraire, vous devrez aller chercher
les sources et les recompiler sur votre machine\footnote{
{\tt ftp://ftp.ibp.fr/pub/gnu} par exemple.}. Cet {\'e}diteur fonctionne
aussi sur d'autres syst{\`e}mes d'exploitation comme {\MacOS}, {\Windows} et
{\OpenVMS}. "{\tt emacs}" est beaucoup plus facile d'approche que
"{\tt vi}". Toutefois, une utilisation pouss{\'e}e de cet {\'e}diteur montre
que, lui aussi, n{\'e}cessite un apprentissage d'un nombre impressionnant de
s{\'e}quence de touches\footnote{les mauvaises langues diront qu'"{\tt
emacs}" repr{\'e}sente les initialises de \esckey, \key{{\sc meta}}, \altkey,
\ctrlkey et \shiftkey.}

"{\tt emacs}" est enti{\`e}rement reprogrammable. En effet, il s'appuie
sur un moteur Lisp\footnote{Le langage Lisp {\'e}tait tr{\`e}s utilis{\'e} dans les
d{\'e}veloppements en intelligence artificiel dans la fin des ann{\'e}es 80.}.
On peut donc d{\'e}velopper toutes les extensions que l'on d{\'e}sire gr{\^a}ce {\`a} ce
langage. "{\tt emacs}" se rapprocherait donc de l'{\'e}diteur "{\tt
LSE}"\footnote{DEC Language Sensitive Editor, disponible sous {\OpenVMS}.}
et du langage associ{\'e} "{\tt TPU}"\footnote{DEC Text Processing
Utility, disponible sous {\OpenVMS}.}. De nombreuses extensions ont {\'e}t{\'e}
r{\'e}alis{\'e}es pour "{\tt emacs}", {\`a} un point tel que certaines personnes
s'en servent comme un environnement de travail complet (gestionnaire de
fichiers, logiciel de messagerie, navigateur Web, environnement de
d{\'e}veloppement, etc.). Il existe une version all{\'e}g{\'e}e d'"{\tt emacs}"
sans le moteur Lisp: "{\tt micro-emacs}".

\begin{remarque}
Il est possible de se programmer un environnement d'{\'e}dition avec "{\tt vi}",
mais il n'est pas programmable au sens o{\`u} on l'entend pour l'{\'e}diteur
"{\tt emacs}".
\end{remarque}

\begin{remarque}
Il existe d'autres {\'e}diteurs de texte sous {\Unix}, mais moins r{\'e}pandus
ou bien sp{\'e}cifique {\`a} un constructeur comme~:
\begin{itemize}
	\item	{\tt textedit} {\'e}diteur de texte livr{\'e} en standard
			sur les machines livr{\'e}es avec SunSolaris et offrant les
			fonctions de base d'un {\'e}diteur de texte.
	\item	{\tt jot} {\'e}diteur de texte livr{\'e} en standard
			sur les machines livr{\'e}es avec Irix de Silicon Graphics et offrant les
			fonctions de base d'un {\'e}diteur de texte.
	\item	{\tt dtpad}, {\'e}diteur de texte offrant les fonctions de base, livr{\'e}
			avec l'environnement CDE\footnote{Common Desktop Environment,
			environnement utilisateur normalis{\'e} par l'OSF, afin d'avoir un
			environnement identique entre constructeurs de stations de travail.
			Actuellement, cet environnement est disponible sur tous les
			{\Unix}s du march{\'e} et m{\^e}me sur des syst{\`e}mes propri{\'e}taires comme
			{\OpenVMS} de Digital Corp.},
	\item	etc.
\end{itemize}
\end{remarque}
