%%%%%%%%%%%%%%%%%%%%%%%%%%%%%%%%%%%%%%%%%%%%%%%%%%%%%%%%%%%%%%%%%%%%%%%%
%                                                                      %
% This program is free software; you can redistribute it and/or modify %
% it under the terms of the GNU General Public License as published by %
% the Free Software Foundation; either version 2 of the License, or    %
% (at your option) any later version.                                  %
%                                                                      %
% This program is distributed in the hope that it will be useful,      %
% but WITHOUT ANY WARRANTY; without even the implied warranty of       %
% MERCHANTABILITY or FITNESS FOR A PARTICULAR PURPOSE.  See the        %
% GNU General Public License for more details.                         %
%                                                                      %
% You should have received a copy of the GNU General Public License    %
% along with this program; if not, write to the Free Software          %
% Foundation, Inc., 51 Franklin St, Fifth Floor, Boston,               %
% MA  02110-1301  USA                                                  %
%                                                                      %
%%%%%%%%%%%%%%%%%%%%%%%%%%%%%%%%%%%%%%%%%%%%%%%%%%%%%%%%%%%%%%%%%%%%%%%%
%
%	$Id$
%


\setcounter{remarque-cnt}{1}
\setcounter{example-cnt}{1}
\chapter{Les arguments des programmes shell}

Tout programme shell peut {\^e}tre invoqu{\'e} depuis la ligne de
commande. Les \index{shell!arguments}arguments de la ligne de commande
sont r{\'e}f{\'e}renc{\'e}s par le programme shell lui-m{\^e}me en
fonction de leur position relative dans la ligne de commande. Ces
arguments sont appel{\'e}s param{\`e}tres ou
\index{variable!positionnelle}variables positionnels. Ils prennent la
valeur correspondante {\`a} l'argument de la ligne de commande. Les
param{\`e}tres positionnels peuvent {\^e}tre utilis{\'e}s dans les
programmes shell de la m{\^e}me mani{\`e}re que les variables shell,
c'est-{\`a}-dire que pour les r{\'e}f{\'e}rencer, il suffit d'utiliser
le symbole "\verb=$=". On ne r{\'e}f{\'e}rencer que jusqu'{\`a} neuf
arguments de la ligne de commande (\verb=$1= {\`a} \verb=$9=). Nous
verrons qu'il est possible de r{\'e}cup{\'e}rer tous les arguments de la
ligne de commande.

L'argument z{\'e}ro de la ligne de commande (variable "\verb=$0=" correspond au
nom de la commande.

\begin{definition}{En r{\'e}sum{\'e}}
Ligne de commande~:\\
\verb=$ nom_prog arg1 arg2 arg3 arg4 ...=

Dans le programme Shell~:\\

\begin{tabular}{|l|l|l|l|}
		\hline
		\multicolumn{2}{|l|}{}					&
		\multicolumn{2}{|c|}{\'{E}quivalences}	\\
	\hline
		\multicolumn{1}{|c|}{Commande shell}	&
		\multicolumn{1}{|c|}{Sortie}			&
		\multicolumn{1}{|c|}{{\OpenVMS}}		&
		\multicolumn{1}{|c|}{Langage C}			\\
	\hline \hline
		{\tt echo \$0}					&
		{\tt nom\_prog}					&
		{\sl pas d'{\'e}quivalent}			&
		{\tt argv[0]}					\\
	\hline
		{\tt echo \$1}					&
		{\tt arg1}						&
		{\tt P1}						&
		{\tt argv[1]}					\\
	\hline
		{\tt echo \$2}					&
		{\tt arg2}						&
		{\tt P2}						&
		{\tt argv[2]}					\\
	\hline
		\multicolumn{4}{|l|}{$\cdots$}	\\
	\hline
		{\tt echo \$8}					&
		{\tt arg8}						&
		{\tt P8}						&
		{\tt argv[8]}					\\
	\hline
		{\tt echo \$9}					&
		{\tt arg9}						&
		{\sl pas d'{\'e}quivalent}			&
		{\tt argv[9]}					\\
	\hline
\end{tabular}
\end{definition}

Il est possible de r{\'e}f{\'e}rencer la totalit{\'e} des arguments
{\`a} l'int{\'e}rieur d'un script shell. Pour cela, la variable
\index{variable!\#@\texttt{\#}}"\verb=$#=" et la commande
\index{shift@\texttt{shift}}"{\tt shift}" seront utilis{\'e}es.
Reportez-vous aux sections \ref{variables-special-var} et
\ref{advcmds-shift}.
