%%%%%%%%%%%%%%%%%%%%%%%%%%%%%%%%%%%%%%%%%%%%%%%%%%%%%%%%%%%%%%%%%%%%%%%%
%                                                                      %
% This program is free software; you can redistribute it and/or modify %
% it under the terms of the GNU General Public License as published by %
% the Free Software Foundation; either version 2 of the License, or    %
% (at your option) any later version.                                  %
%                                                                      %
% This program is distributed in the hope that it will be useful,      %
% but WITHOUT ANY WARRANTY; without even the implied warranty of       %
% MERCHANTABILITY or FITNESS FOR A PARTICULAR PURPOSE.  See the        %
% GNU General Public License for more details.                         %
%                                                                      %
% You should have received a copy of the GNU General Public License    %
% along with this program; if not, write to the Free Software          %
% Foundation, Inc., 51 Franklin St, Fifth Floor, Boston,               %
% MA  02110-1301  USA                                                  %
%                                                                      %
%%%%%%%%%%%%%%%%%%%%%%%%%%%%%%%%%%%%%%%%%%%%%%%%%%%%%%%%%%%%%%%%%%%%%%%%
%
%	$Id$
%

%%%%%%%%%%%%%%%%%%%%%%%%%%%%%%%%%%%%%%%%%%%%%%%%%%%%%%%%%%%%%%%%%%%%%
% Conventions
%
\section{Conventions et Notations}

Dans toute la suite de ce document, les conventions suivantes seront adopt{\'e}es~:
\begin{description}
	\item[{\tt \% commande}]\mbox{}\\
		repr{\'e}sente la commande saisie par l'utilisateur au ni\-veau de l'invite
		({\sl prompt}) {\Unix}.\\
		Cette commande ob{\'e}ira aux r{\`e}gles explicit{\'e}es au niveau de la section
		\ref{bcpts-formcmd}.
	\item[{\rm <<~$\sqcup$~>> ou \spacekey }]\mbox{}\\
		repr{\'e}sente le caract{\`e}re <<~{\sl espace}~>>.
	\item[\tabkey]\mbox{}\\
		repr{\'e}sente le caract{\`e}re <<~{\sl Tabulation}~>>.
	\item[\altkey]\mbox{}\\
		repr{\'e}sente la touche <<~{\sl ALT}~>>.
		En g{\'e}n{\'e}ral, la touche <<~{\sl ALT}~>> se trouve de part et d'autre de la
		barre d'espace du clavier alpha-num{\'e}rique sur les claviers {\'e}tendus 102 touches.
	\item[\shiftkey]\mbox{}\\
		repr{\'e}sente la touche <<~{\sl SHIFT}~>> ou <<~$\uparrow$~>>,
		ou encore <<~{\sl Maj}~>>. Cette touche permet de passer en mode
		<<~{\sl majuscule}~>>.\\
		En g{\'e}n{\'e}ral, la touche <<~{\sl SHIFT}~>> se trouve de part et d'autre
		du clavier alpha-num{\'e}rique sur les claviers {\'e}tendus 102 touches.
	\item[\ctrlkey]\mbox{}\\
		repr{\'e}sente la touche <<~{\sl CONTROL}~>> ou <<~{\sl CTRL}~>>,
		ou encore <<~{\sl ctrl}~>>.\\
		En g{\'e}n{\'e}ral, la touche <<~{\sl CTRL}~>> se trouve de part et d'autre en bas
		du clavier alpha-num{\'e}rique sur les claviers {\'e}tendus 102 touches.
	\item[\returnkey]\mbox{}\\
		repr{\'e}sente la touche <<~{\sl RETURN}~>> ou <<~{\sl ENTR\'{E}E}~>>
		ou encore <<~\fbox{$\hookleftarrow$}~>>.\\
		En g{\'e}n{\'e}ral, la touche <<~{\sl RETURN}~>> se trouve {\`a} droite
		du clavier alpha-num{\'e}rique.
	\item[\esckey]\mbox{}\\
		repr{\'e}sente la touche <<~{\sl ESCAPE}~>> ou <<~{\sl esc}~>>.\\
		En g{\'e}n{\'e}ral, la touche {\sl ESCAPE} se trouve en haut {\`a} gauche
		sur les claviers {\'e}tendus 102 touches.
	\item[\key{{\tt x}}]\mbox{}\\
		repr{\'e}sente la touche du clavier permettant d'obtenir le caract{\`e}re
		<<~{\tt x}~>>.
	\item[\key{{\tt X}}]\mbox{}\\
		repr{\'e}sente la combinaison des touches \shiftkey et
		\key{{\tt x}}.
	\item[\seqkey{{\sl X}}{{\sl Y}}]\mbox{}\\
		repr{\'e}sente l'appuie simultan{\'e} sur les touches <<~{\sl X}~>> et
		<<~{\sl Y}~>> du clavier.
\end{description}

%%%%%%%%%%%%%%%%%%%%%%%%%%%%%%%%%%%%%%%%%%%%%%%%%%%%%%%%%%%%%%%%%%%%%
% \section*{Correspondance avec les claviers <<~{\sl LK200}~>>
%	et <<~{\sl LK400}~>>}
%
% Les claviers <<~{\sl LK200}~>> et <<~{\sl LK400}~>> sont utilis{\'e}s sur
% les terminaux de type <<~{\tt VT200}~>>, <<~{\tt VT300}~>>, <<~{\tt VT400}~>> et
% <<~{\tt VT500}~>> de Digital Equipment Corp. Ils poss{\`e}dent vingt touches de
% fonction et un pav{\'e} num{\'e}rique l{\'e}g{\`e}rement diff{\'e}rent des claviers standard
% 102 touches. Ils disposent des touches \fbox{{\tt CTRL}}.
%
% Les claviers <<~{\sl LK200}~>> fonctionnent sous {\Unix} avec
% les possibilit{\'e}s suivantes~:
% \begin{quote}
% \begin{center}
% \begin{tabular}{|@{\hspace{0.5ex}}c@{\hspace{0.5ex}}|c|p{6cm}|}
%	\hline
%		Touche				&	Disponible	&
%			Substitution	\\
%	\hline \hline
%		\ctrlkey			&	Oui			&
%							\\[1ex]
%		\altkey				&	Non			&
%			Aucune substitution possible.		\\[1ex]
%		\esckey				&	Non			&
%			\key{{\sc f12}}, \control{{\tt 3}},
%			ou 	\control{$]$}	\\
%	\hline
% \end{tabular}
% \end{center}
% \end{quote}
%
% Les claviers <<~{\sl LK400}~>> fonctionnent sous {\Unix} avec
% les possibilit{\'e}s suivantes~:\\
% \begin{quote}
% \begin{center}
% \begin{tabular}{|@{\hspace{0.5ex}}c@{\hspace{0.5ex}}|c|p{6cm}|}
%	\hline
%		Touche				&	Disponible		&
%			Substitution	\\
%	\hline \hline
%		\ctrlkey			&	Oui				&
%							\\[1ex]
%		\altkey				&	Oui				&
%							\\[1ex]
%		\esckey				&	Param{\`e}trable	&
%			Vient en substition de la touche \key{$\tilde{}$}/\key{`}.
%			Si la fonction \esckey est valid{\'e}e sur le clavier, les
%			caract{\`e}res <<~$\tilde{}$~>> et <<~`~>> sont accessibles
%			{\`a} partir de la touche \key{$>$}/\key{$<$} en bas {\`a} gauche
%			du clavier alphanum{\'e}rique. Si cette fonction 
%			n'est pas valid{\'e}e, la configuration est identique {\`a} celle
%			des claviers <<~{\sl LK200}~>>, c'est-{\`a}-dire
%			\key{{\sc f12}}, \control{{\tt 3}},
%			ou 	\control{$]$}	\\
%	\hline
% \end{tabular}
% \end{center}
% \end{quote}
