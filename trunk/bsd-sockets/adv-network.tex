%%%%%%%%%%%%%%%%%%%%%%%%%%%%%%%%%%%%%%%%%%%%%%%%%%%%%%%%%%%%%%%%%%%%%
%
% Notions avanc{\'e}es de r{\'e}seau sur les syst{\`e}mes Unix
%
\chapter[Notions avanc{\'e}es de r{\'e}seau]
	{\label{advnet}Notions avanc{\'e}es de r{\'e}seau sur les syst{\`e}mes {\Unix}}

%%%%%%%%%%%%%%%%%%%%%%%%%%%%%%%%%%%%%%%%%%%%%%%%%%%%%%%%%%%%%%%%%%%%%
\section{\label{advnet-inetd}Service <<~\texttt{inetd}~>>}

%%%%%%%%%%%%
\subsection{Introduction - Fonctionnalit{\'e}}

Nous avons vu qu'un processus serveur est un processus qui doit~:
\begin{itemize}
	\item	{\'e}couter sur un port de communication pour attendre les
		demandes de connexion dans le cas des {\'e}changes en mode
		connect{\'e}, ou bien attendre en permance que des donn{\'e}es
		lui soient transmises, dans le cas des {\'e}changes de donn{\'e}es
		en mode non-connect{\'e},
	\item	{\'e}changer des informations avec un ou plusieurs clients.
\end{itemize}

Par cons{\'e}quent, si vous d{\'e}sirez que le service soit disponible en
permance, {\bf le processus doit toujours {\^e}tre actif}. Ce point pr{\'e}sente
donc les inconv{\'e}nients suivants~:
\begin{itemize}
	\item	ce processus va consommer des ressources machines inutilement
		si les acc{\`e}s aux services qu'il propose sont peu sollicit{\'e}s,
	\item	si l'on doit disposer de beaucoup de services sur une m{\^e}me
		machine, il risquera d'y avoir profusions de processus.
\end{itemize}

La technique consiste donc d'avoir un {\sl serveur de services r{\'e}seau},
c'est-{\`a}-dire un processus qui est {\`a} l'{\'e}coute sur un ensemble de ports
donn{\'e}es. D{\`e}s qu'une demande de connexion arrive sur l'un d'eux (pour les
{\'e}changes de donn{\'e}es en mode connect{\'e}) ou bien que des donn{\'e}es y sont
disponibles (pour les {\'e}changes de donn{\'e}es en mode non-connect{\'e}), ce dernier
va cr{\'e}er un sous processus dans lequel il ex{\'e}cutera le serveur associ{\'e}~:
{\bf c'est le r{\^o}le du service <<~\texttt{inetd}~>>}\footnote{<<~\texttt{inetd}~>> =Internet Daemon}

<<~\texttt{inetd}~>> est donc un processus tournant en permanence sur le syst{\`e}me
{\Unix}\footnote{pour les implantations des services IP sur d'autre syst{\`e}mes
d'exploitation, on retrouvera le m{\^e}me principe, c'est-{\`a}-dire un processus
qui tournera en permance pour {\^e}tre {\`a} l'{\'e}coute sur une liste r{\'e}f{\'e}renc{\'e}e de ports
de communication}\footnote{un processus tournant en permanence sous {\Unix} est
appel{\'e} aussi {\sl daemon}}. Il sera alors responsable de d{\'e}marrer des services
classiques comme certains services ARPA, Berkeley et NFS ou bien ceux que vous
auriez choisi de d{\'e}velopper, lorsque l'un d'eux sera sollicit{\'e}. <<~\texttt{inetd}~>>
pourra prendre en charge les deux types de connexion au niveau des serveurs~:
\begin{itemize}
	\item	les serveurs utilisant un mode de communication non-connect{\'e} en
		s'appuyant sur le protocole de transport ad{\'e}quat (en g{\'e}n{\'e}ral UDP)
	\item	les serveurs utilisant un mode de communication connect{\'e} en
		s'appuyant sur le protocole de transport ad{\'e}quat (en g{\'e}n{\'e}ral TCP)
\end{itemize}

Les �tapes suivies par <<~\texttt{inetd}~>> sont les suivantes~:
\begin{order}
	\item	<<~\texttt{inetd}~>> est en {\'e}coute sur tous les ports de communication
		d{\'e}crits dans son fichier de configuration,
	\item	lorsqu'une demande de connexion est re\c{c}ue sur l'un des ports,
		<<~\texttt{inetd}~>> examine son fichier de configuration afin de conna{\^\i}tre
		le serveur (programme) {\`a} lancer,
	\item	<<~\texttt{inetd}~>> va cr{\'e}er un sous-process\footnote{appel syst{\`e}me
		{\tt fork()}} dans lequel il  ex{\'e}cutera le programme serveur
		ad{\'e}quate,
	\item	le programme serveur h{\'e}ritera des canaux d'entr{\'e}e$/$sortie
		via les canaux classiques {\tt stdin} et {\tt stdout}:
		\begin{itemize}
			\item	{\tt stdin} correspondra au canal par lequel
				arriveront les donn{\'e}es,
			\item	{\tt stdout} correspondra au canal par lequel
				les donn{\'e}es devront {\^e}tre {\'e}mises.
		\end{itemize}
\end{order}

Pendant toute cette phase de lancement du serveur ad�quat,
<<~\texttt{inetd}~>> continuera {\`a} {\'e}couter sur tous les ports de communication
dont il a la charge. Par cons{\'e}quent, pendant cette phase, il est tout {\`a}
fait possible d'effectuer les m{\^e}mes �tapes pour~:
\begin{itemize}
	\item	un autre service,
	\item	le m{\^e}me service.
\end{itemize}

\begin{remarque}
Sachant qu'<<~\texttt{inetd}~>> prend en charge l'{\'e}tablissement de connexion,
que ce soit pour les ports de communication en mode connect{\'e} ou
non-connect{\'e}, il ne sera pas n{\'e}cessaire d'utiliser~:
\begin{itemize}
	\item	{\tt socket()}, {\tt bind()}, {\tt listen()} et {\tt accept()}
		pour la gestion du canal en mode connect{\'e},
	\item	{\tt socket()} et {\tt bind()} pour la gestion du canal
		en mode non-connect{\'e}.
\end{itemize}
Il suffira de lire et d'{\'e}crire respectivement sur l'entr{\'e}e et la sortie standard.
\end{remarque}

\begin{remarque}
<<~\texttt{inetd}~>> n'a d'int{\'e}r{\^e}t que pour les processus serveur. Il n'est
pas con\c{c}u pour g{\'e}rer des programmes clients, ce n'est d'ailleurs pas
le but recherch{\'e}.
\end{remarque}

%%%%%%%%%%%%
\subsection{\label{advnet-config-inetd}Configuration}

La configuration d'<<~\texttt{inetd}~>> sur un syst{\`e}me {\Unix} ne peut
se faire qu'{\`a} partir du compte administrateur <<~{\tt root}~>>. Vous
devez donc avoir acc{\`e}s {\`a} ce compte sp{\'e}cial pour ajouter ou retirer des
services, ou bien demander {\`a} l'administrateur de votre machine d'en faire le param�trage.

Avant de vouloir ajouter une entr{\'e}e dans <<~\texttt{inetd}~>>, il faut s'assurer
que ce service est bien d{\'e}clar{\'e} dans la liste des services r{\'e}f{\'e}renc{\'e}s,
c'est-{\`a}-dire dans le fichier {\tt $/$etc$/$services} sur un syst{\`e}me
{\Unix} (cf. section \ref{protip-etc-services}). Comme pour <<~\texttt{inetd}~>>, l'ajout ou le retrait
de services ne peut se faire qu'{\`a} partir du compte d'administration de la machine.

De m{\^e}me, il faudra s'assurer que le protocole de transport {\`a} utiliser,
est bien d{\'e}clar{\'e} dans la liste des protocoles r{\'e}f{\'e}renc{\'e}s, c'est-{\`a}-dire
dans le fichier {\tt $/$etc$/$protocols} sur un syst{\`e}me {\Unix} (cf.
section \ref{protip-etc-protocols}). Comme pour <<~\texttt{inetd}~>>, l'ajout ou
le retrait d'un protocole ne peut se faire qu'{\`a} partir du compte d'administration de la machine.

Le fichier de configuration d'<<~\texttt{inetd}~>> est {\tt $/$etc$/$inetd.conf}.
Celui-ci se compose de six champs s{\'e}par{\'e}s par des tabulations ou des espaces~:
\begin{description}
	\item[Le nom du service]\mbox{}\\
		Ce champ correspond {\`a} un nom de service
		qui a {\'e}t{\'e} enregistr{\'e} dans le fichier {\tt $/$etc$/$services}
		(ou dans l'annuaire correspondant).
	\item[Le type de socket]\mbox{}\\
		Ce champ correspond au mode de connexion
		utilis{\'e}, c'est-{\`a}-dire mode connect{\'e} ou mode non-connect{\'e}.
		Les options valides sont~:\\
		\begin{tabular}{l@{\hspace{2ex}}p{10cm}}
			{\tt stream}	&	pour les sockets en mode connect{\'e},	\\
			{\tt dgram}	&	pour les sockets en mode non-connect{\'e}.	\\
		\end{tabular}
	\item[Le protocole de transport]\mbox{}\\
		Ce champ sp{\'e}cifie le protocole de transport {\`a} utiliser.
		Les noms valides sont ceux qui auront {\'e}t{\'e} d{\'e}clar{\'e}s dans le fichier
		{\tt $/$etc$/$protocols} ou dans l'annuaire {\'e}quivalent. On pourra donc avoir~:
		\begin{itemize}
			\item <<~{\tt tcp}~>> pour le protocole de transport TCP,
			\item <<~{\tt udp}~>> pour le protocole de transport UDP.
		\end{itemize}
		{\bf ATTENTION}, le protocole de transport que vous
		aurez sp{\'e}cifi{\'e}, doit {\^e}tre capable de g{\'e}rer le mode de
		communication que vous aurez s{\'e}lectionn{\'e} au niveau du champ pr{\'e}c{\'e}dent.
	\item[Les options de lancement]\mbox{}\\
		Ce champ admet deux valeurs possibles~: <<~{\tt wait}~>> et <<~{\tt nowait}~>>. Cette option n'a de sens
		que dans le cas des ports de communications fonctionnant
		en mode non-connect{\'e} ({\sl datagram sockets}). Dans tous les
		autres cas, la seule option possible est <<~{\tt nowait}~>>.\\
		Pour les ports de communication fonctionnant en mode non connect{\'e}, on a donc deux cas possibles.
		\begin{description}
			\item[Avec l'option {\tt wait}]\mbox{}\\
				le serveur {\'e}changera des donn{\'e}es avec un client,
				puis lib{\`e}rera le port de communication. <<~\texttt{inetd}~>>
				pourra alors recevoir d'autres messages sur ce m{\^e}me
				port~: le serveur est dit alors <<~{\sl multi-threaded}~>>.
			\item[Avec l'option {\tt nowait}]\mbox{}\\
				Le serveur recevra les donn{\'e}es de plusieurs sessions
				sur le m{\^e}me port de communication. Ce type de serveur est dit
				<<~{\sl single-threaded}~>>.
		\end{description}
		En r{\'e}sum{\'e}~:
		\begin{itemize}
			\item	Avec l'option {\tt wait}, le serveur doit terminer
				le traitement en cours pour prendre en charge une autre requ{\^e}te.
			\item	Avec l'option {\tt nowait}, plusieurs requ{\^e}tes peuvent
				{\^e}tre trait{\'e}es simultan{\'e}ment.
		\end{itemize}
	\item[Le nom de l'utilisateur]\mbox{}\\
		Ce champ pr{\'e}cise sous quelle identit{\'e} le processus serveur doit {\^e}tre lanc{\'e}.
	\item[Le nom du programme serveur]\mbox{}\\
		Ce champ se compose de deux parties~:
		\begin{itemize}
			\item	la premi{\`e}re consiste {\`a} sp{\'e}cifier le chemin d'acc{\`e}s 
				absolu du programme {\`a} lancer, c'est-{\`a}-dire le nom du
				r{\'e}pertoire depuis la racine suivi du nom de l'ex{\'e}cutable,
			\item	la seconde consiste {\`a} sp{\'e}cifier {\`a} nouveau le nom du
				programme, {\bf uniquement le nom et non pas son
				chemin d'acc{\`e}s}, et les {\'e}ventuels arguments {\`a} lui 	pr{\'e}ciser.
		\end{itemize}
		Cette contrainte de pr{\'e}ciser une fois le chemin absolu et
		ensuite le nom du programme suivi de ces arguments, s'explique
		si l'on fait l'analogie avec le langage C~:
		\begin{itemize}
			\item	le chemin absolu permet {\`a} <<~\texttt{inetd}~>> de localiser les informations et
				ainsi de pouvoir charger le bon ex{\'e}cutable,
			\item	repr{\'e}ciser le nom de l'ex{\'e}cutable sans le chemin d'acc�s
				suivi des arguments, permet d'initialiser les variables <<~{\tt argc}~>> et
				<<~{\tt argv}~>> d'un programme C. Par cons{\'e}quent, la variable
				<<~{\tt argv[0]}~>> contiendra uniquement le nom de l'ex{\'e}cutable sans
				aucun nom de r{\'e}pertoire dans son chemin d'acc{\`e}s.
		\end{itemize}
\end{description}

Une fois que le fichier {\tt $/$etc$/$inetd.conf} a {\'e}t{\'e} modifi{\'e}, il
faut pr{\'e}venir <<~\texttt{inetd}~>> de le relire pour prendre en compte la mise {\`a}
jour. \'{E}videmment, cette op{\'e}ration ne peut s'effectuer qu'{\`a} partir 
du compte d'administration de la machine. Suivant les {\Unix}, on trouvera plus ou moins de facilit{\'e}s.
Par exemple~:
\begin{itemize}
	\item	sur HP-UX\footnote{L'{\Unix} de Hewlett-Packard}, il suffit de taper la commande <<~{\tt inetd -c}~>>,
	\item	sur {\sc Irix}\footnote{L'{\Unix} de Silicon Graphics} et {\Linux}, il suffit de taper la commande
		<<~{\tt killall -HUP inetd}~>>,
	\item	sur AIX\footnote{L'{\Unix} d'IBM}, il suffit de taper la commande <<~{\tt refresh -s inetd}~>>,
	\item	etc.
\end{itemize}
De fa\c{c}on g{\'e}n{\'e}rale, il suffit d'envoyer le signal <<~{\tt SIGHUP}~>> au
processus <<~\texttt{inetd}~>>. On peut donc le faire avec la simple commande suivante~:
\begin{quote}
{\tt kill -HUP} {\it PID}
\end{quote}
o{\`u} <<~{\it PID}~>> repr{\'e}sente le num{\'e}ro de processus de <<~\texttt{inetd}~>>.{\bf  Ceci
fonctionnera sur n'importe quel syst{\`e}me {\Unix}}.

\begin{remarque}
Contrairement {\`a} la liste des services disponibles ({\tt $/$etc$/$services})
le fichier {\tt $/$etc$/$inetd.conf} ne peut {\^e}tre que local {\`a} une machine.
Vous ne pouvez donc pas le g{\'e}rer via des annuaires r{\'e}partis comme
le service NIS.
\end{remarque}

%%%%%%%%%%%%
\subsection{Aspect administration de syst{\`e}mes}

<<~\texttt{inetd}~>> peut pr{\'e}sent{\'e} de nombreux avantages quant {\`a} l'aspect
<<~gestion de syst{\`e}mes~>>, plus particuli{\`e}rement avec {\Unix}.

Par exemple, il apparait {\'e}vident que des serveurs comme
\begin{itemize}
	\item	{\tt telnetd} assurant la gestion des connexions <<~Telnet~>> entrantes,
	\item	{\tt ftpd} assurant la gestion des connexions <<~Ftp~>> entrantes,
	\item	etc.
\end{itemize}
devront {\^e}tre g{\'e}r{\'e}s par <<~\texttt{inetd}~>>. En effet, cela {\'e}vitera d'avoir une
liste bien trop importante de processus en attente. D{\`e}s que cela sera
n{\'e}cessaire, le serveur associ{\'e} sera lanc{\'e}. D{\`e}s que la connexion
sera termin{\'e}e, le processus dispara{\^\i}tra.

Par contre, pour des processus serveurs comme celui d'un serveur {\sc Web},
cela risque de ne pas {\^e}tre une bonne id{\'e}e. En effet, pour certains types
de services, un ou plusieurs fichiers de configuration
permettent de d{\'e}crire tous les param{\`e}tres du service. De plus, dans
le cas particulier d'un serveur {\sc Web}, {\`a} chaque demande d'une page~:
\begin{itemize}
	\item	le client doit {\'e}tablir une connexion au serveur,
	\item	le serveur doit envoyer les fichiers correspondants,
	\item	la connexion entre les deux est ferm{\'e}e.
\end{itemize}

Par cons{\'e}quent, si le serveur {\sc Web} est g{\'e}r{\'e} par <<~\texttt{inetd}~>>,
{\`a} chaque connexion d'un client
\begin{quote}
\begin{tabular}{cl}
	{\bf {\'e}tape 1}	&	le processus sera lanc{\'e}, \\
	{\bf {\'e}tape 2}	&	il va lire ses fichiers de configuration, \\
	{\bf {\'e}tape 3}	&	il va transf{\'e}rer les donn{\'e}es, \\
	{\bf {\'e}tape 4}	&	le processus se termine.\\
\end{tabular}
\end{quote}
Par contre, si ce type de serveur tourne en permanence, seules les
{\'e}tapes 2 et 3 seront ex{\'e}cut{\'e}es {\`a} chaque connexion. De plus, certains
se clonent automatiquement en fonction du nombre de requ{\^e}tes qui leur
est adress{\'e}, ceci afin de satisfaire les demandes plus rapidement.

Dans tous les cas, il faudra donc r{\'e}fl{\'e}chir avant de faire prendre
en charge un processus serveur par <<~\texttt{inetd}~>>.

Voici une liste non-exhaustive des services classiques dont la gestion
est assur{\'e}e par <<~\texttt{inetd}~>>~:
\begin{itemize}
	\item	les services de connexion et d'ex{\'e}cution {\`a} distance, 	comme <<~{\tt telnetd}~>>,
		<<~{\tt rlogind}~>>, <<~{\tt rshd}~>>, <<~{\tt rexecd}~>>, etc.
	\item	les services de transfert de fichiers, comme <<~{\tt ftpd}~>>, <<~{\tt rcpd}~>>, etc.
	\item	le service de changement des mots de passe des boites aux lettres associ� au protocole de
		messagerie POP3 (<<~{\tt poppasswd}~>>),
	\item	etc.
\end{itemize}


Voici un exemple de fichier {\tt $/$etc$/$inetd.conf}~:

\begin{verbatim}
# See "man 8 inetd" for more information.
#
# If you make changes to this file, either reboot your machine or send
the
# inetd a HUP signal:
# Do a "ps x" as root and look up the pid of inetd. Then do a
# "kill -HUP <pid of inetd>".
# The inetd will re-read this file whenever it gets that signal.
#
# <service_name> <sock_type> <proto> <flags> <user> <server_path> <args>
#
echo    stream  tcp nowait  root    internal
echo    dgram   udp wait    root    internal
discard stream  tcp nowait  root    internal
discard dgram   udp wait    root    internal
daytime stream  tcp nowait  root    internal
daytime dgram   udp wait    root    internal
chargen stream  tcp nowait  root    internal
chargen dgram   udp wait    root    internal
time    stream  tcp nowait  root    internal
time    dgram   udp wait    root    internal
#
# These are standard services.
#
ftp     stream  tcp nowait  root    /usr/sbin/tcpd  wu.ftpd
telnet  stream  tcp nowait  root    /usr/sbin/tcpd  in.telnetd
#
# Use this one instead if you want to snoop on telnet users (try to use this
# for ethical purposes, ok folks?) :
# telnet stream tcp nowait  root    /usr/sbin/tcpd /usr/sbin/in.telnetsnoopd
#
# If you want to read NNTP news via TERM, comment out the nntp
# line below, and use a command like this once the TERM
# connection is up:  tredir 119 my.nntp.host:119
# You'll also want to do this:
#    set NNTPSERVER my.nntp.host ; export NNTPSERVER
nntp    stream  tcp nowait  root    /usr/sbin/tcpd  in.nntpd
#
# This is for BSD sendmail:
# smtp  stream  tcp nowait  root    /usr/sbin/tcpd    sendmail -v
# This is set up for running Smail:
# smtp  stream  tcp nowait  root    /usr/sbin/tcpd  /usr/bin/rsmtp -bs
#
# The comsat daemon notifies the user of new mail when biff is set to y:
comsat  dgram   udp wait    root    /usr/sbin/tcpd  in.comsat
#
# Shell, login, exec and talk are BSD protocols.
#
shell   stream  tcp nowait  root    /usr/sbin/tcpd  in.rshd -L
login   stream  tcp nowait  root    /usr/sbin/tcpd  in.rlogind
# exec  stream  tcp nowait  root    /usr/sbin/tcpd  in.rexecd
# talk  dgram   udp wait    root    /usr/sbin/tcpd  in.talkd
ntalk   dgram   udp wait    root    /usr/sbin/tcpd  in.talkd
#
# Kerberos authenticated services
#
# klogin    stream  tcp nowait  root    /usr/sbin/tcpd  rlogind -k
# eklogin   stream  tcp nowait  root    /usr/sbin/tcpd  rlogind -k -x
# kshell    stream  tcp nowait  root    /usr/sbin/tcpd  rshd -k
#
# Services run ONLY on the Kerberos server
#
# krbupdate stream  tcp nowait  root    /usr/sbin/tcpd  registerd
# kpasswd   stream  tcp nowait  root    /usr/sbin/tcpd  kpasswdd
#
# Pop et al
#
# pop2  stream  tcp nowait  root    /usr/sbin/tcpd  in.pop2d
pop3    stream  tcp nowait  root    /usr/sbin/tcpd  in.pop3d
#
# The Internet UUCP service.
#
# uucp  stream  tcp nowait  uucp    /usr/sbin/tcpd  /usr/lib/uucp/uucico
-l
#
# Tftp service is provided primarily for booting.  Most sites
# run this only on machines acting as "boot servers." 
#
# tftp   dgram  udp wait    nobody  /usr/sbin/tcpd  in.tftpd
# bootps dgram  udp wait    root    /usr/sbin/in.bootpd in.bootpd
#
# Finger, systat and netstat give out user information which may be
# valuable to potential "system crackers."  Many sites choose to 

# disable some or all of these services to improve security.
# Try "telnet localhost systat" and "telnet localhost netstat" 
# to see that information yourself!
# 
finger  stream  tcp nowait  nobody  /usr/sbin/tcpd  in.fingerd -w
systat  stream  tcp nowait  nobody  /usr/sbin/tcpd  /bin/ps -auwwx
netstat stream  tcp nowait  root    /usr/sbin/tcpd  /bin/netstat -a
#
# Ident service is used for net authentication
auth    stream  tcp nowait  root    /usr/sbin/in.identd in.identd
#
# These are to start Samba, an smb server that can export filesystems to
# Pathworks, Lanmanager for DOS, Windows for Workgroups, Windows95, 
# Lanmanager for Windows, Lanmanager for OS/2, Windows NT, etc.  
# Lanmanager for dos is available via ftp from ftp.microsoft.com in 
# bussys/MSclient/dos/. Please read the licensing stuff before
downloading.
# Use the TCP/IP option in the client. Add your server to the \etc\lmhosts
# (or equivalent) file on the client.
netbios-ssn     stream  tcp     nowait  root    /usr/sbin/smbd  smbd
netbios-ns      dgram   udp     wait    root    /usr/sbin/nmbd  nmbd
#
# Sun-RPC based services.
# <service name/version><sock_type><rpc/prot><flags><user><server><args>
#
# rstatd/1-3    dgram   rpc/udp wait    root    /usr/sbin/tcpd 
rpc.rstatd
# rusersd/2-3   dgram   rpc/udp wait    root    /usr/sbin/tcpd 
rpc.rusersd
# walld/1       dgram   rpc/udp wait    root    /usr/sbin/tcpd 
rpc.rwalld
#
# End of inetd.conf.
\end{verbatim}

%%%%%%%%%%%%
\subsection{Extended Internet Daemon <<~{tt xinetd}~>>}

Plusieurs {\Unix}, dont {\Linux} disposent d'une variante de \texttt{inetd}~: \texttt{xinetd}.

Les fonctionnalit�s sont identiques � celles explicit�es pour \texttt{inetd}. La seule
diff�rence r�sident quant aux possibilit�s d'administration. Celles-ci reponsent
essentiellement sur des aspects s�curit�. En effet, il sera possible~:
\begin{itemize}
	\item	d'offrir un service r�seau � une ou plusieurs machines bien identifi�es, 
	\item	de disposer de traces syst�mes (fichiers <<~log~>>) distincts pour chaque service,
	\item	etc.
\end{itemize}

La configuration de <<~\texttt{xinetd}~>> est totalement diff�rente de <<~\texttt{inetd}~>>. Nous n'aborderons
pas ici ce sujet, seul l'aspect fonctionnel nous interesse.

Dans toute la suite, nous n'�voquerons qu'<<~\texttt{inetd}~>>, le comportement de <<~\texttt{xinetd}~>>
�tant identique.

%%%%%%%%%%%%
\subsection{Programmation avec <<~\texttt{inetd}~>>}

Dans cette section, nous allons uniquement donner le source d'un serveur et d'un client utilisant
un canal de communication en mode non-connect{\'e}.

La gestion du serveur est assur{\'e}e par <<~\texttt{inetd}~>>. Le client a un fonctionnement normal.

On supposera que le service est d{\'e}clar{\'e} dans le fichier de configuration
d'<<~\texttt{inetd}~>>. De m{\^e}me, on supposera que le service est bien d{\'e}clar{\'e}
dans le fichier {\tt $/$etc$/$services} que ce soit au niveau de la machine cliente que de la
machine serveur (cf. section \ref{protip-etc-services}). Le service utilis{\'e} s'appellera
<<~{\tt udpexemple}~>>.

{\bf Rappel~:}
\begin{quote}
\begin{itemize}
	\item	l'identifiant de canal {\tt 0} correspond {\`a} l'entr{\'e}e standard {\tt stdin},
	\item	l'identifiant de canal {\tt 1} correspond {\`a} la sortie standard {\tt stdout},
	\item	l'identifiant de canal {\tt 2} correspond {\`a} la sortie d'erreurs standard {\tt stderr}.
\end{itemize}
\end{quote}

\subsubsection{Programme serveur}

\begin{verbatim}
#include <sys/types.h>
#include <sys/socket.h>
#include <netinet/in.h>
#include <stdio.h>
#define  BUFLEN 2000

main ()
{
    int                sock, length, n;
    struct sockaddr_in sock_addr;
    char               buff[BUFLEN];

    length = sizeof(sock_addr);

    if ( (n=recvfrom (0,buf, BUFLEN, 0, &sock_addr, &length))
            < 0 ) {
        perror ("serveur recvfrom()");
        exit (-1);
    }

    if ( sendto (1, buf, n, 0, &sock_addr, length) < 0 ) {
        perror ("serveur sendto()");
        exit (-1);
    }
}
\end{verbatim}

\subsubsection{Programme client}

\begin{verbatim}
#include <sys/types.h>
#include <sys/socket.h>
#include <netinet/in.h>
#include <stdio.h>
#include <netdb.h>

#define BUFLEN 2000
#define DATA "Voici le message � transf�rer en mode non connect�"

main (argc, argv)
int   argc;
char *argv[];
{
    int                 sock;
    struct sockaddr_in  sock_addr, to_addr;
    struct hostent     *host_address;
    struct servent     *sp;
    char                buff[BUFLEN];

    if ( argc != 2 ) {
        fprintf (stderr, "usage: %s hostname\n", argv[0]);
        exit (-1);
    }

    if ( (sock = socket(AF_INET, SOCK_DGRAM, 0)) == -1 ) {
        perror ("client socket()");
        exit (-1);
    }

    /* On va laisser le syst�me attribuer le num�ro de port.
    ** Cela n'est pas g�nant car, lorsqu'on va faire appel
    ** a sendto() sur la socket correspondante, il faudra
    ** donner le num�ro de port. Lorsque le serveur fera un
    ** sendto(), � son tour, il r�cup�rera le num�ro de port
    ** gr�ce � inetd.
    */
    sock_addr.sin_addr.s_addr = INADDR_ANY;
    sock_addr.sin_family      = AF_INET;
    sock_addr.sin_port        = 0;

    if ( bind (sock, &sock_addr, sizeof (sock_addr)) == -1 ) {
        perror ("bind()");
        fprintf (stderr, "%s: erreur avec bind()\n", argv[0]);
        exit (-1);
    }

    if ( (host_address = gethostbyname (argv[1])) < 0 ) {
        fprintf (stderr, "%s: machine %s non d�fini\n",
            argv[0], argv[1]);
        exit (-1);
    }

    if ( (sp = getservbyname ("udpexemple", "udp")) < 0 ) {
        fprintf (stderr, "%s: service udpexemple/udp non d�fini\n",
            argv[0]);
        exit (-1);
    }

    to_addr.sin_addr.s_addr =
        ((struct in_addr *)(host_address->h_addr))->s_addr;
    to_addr.sin_port = sp->s_port;
    to_addr.sin_family = AF_INET;

    if ( sendto(sock, DATA, strlen(DATA), 0, &to_addr,
                sizeof(to_addr)) < 0 ) {
        perror ("client sendto()");
        exit (-1);
    }
    printf ("%s: �mission des donn�es effectu�e\n", argv[0]);

    printf ("%s: d�marrage de la r�ception des donn�es\n", argv[0]);
    if ( read (sock, buf, BUFLEN) < 0 ) {
        perror ("client read()");
        exit (-1);
    }
    printf ("%s: donn�es re�ues \"%s\"\n", argv[0], buf);

    close (sock);
}
\end{verbatim}

%%%%%%%%%%%%%%%%%%%%%%%%%%%%%%%%%%%%%%%%%%%%%%%%%%%%%%%%%%%%%%%%%%%%%
\section{\label{advnet-sockopt-syscall}Examiner et mettre des options sur les sockets}

%%%%%%%%%%%
\subsection{Appels syst{\`e}mes associ{\'e}s}

Il est possible de positionner certaines options sur les canaux
de communication qui ont {\'e}t{\'e} cr{\'e}{\'e}s. De m{\^e}me, on pourra avoir la liste des options.

Ces options permettent de modifier le comportement par d{\'e}faut
des ports de communication, afin de r{\'e}pondre {\`a} des besoins pr{\'e}cis.
Les sections \ref{advnet-sockopt-stream} et \ref{advnet-sockopt-other} 
donnent une liste non exhaustive des options disponibles.

Les appels syst{\`e}mes concern{\'e}s sont~:
\begin{itemize}
	\item	{\tt getsockopt()} permet d'avoir la liste des options
		qui auront {\'e}t{\'e} positionn{\'e}es,
	\item	{\tt setsockopt()} permet de positionner une option.
\end{itemize}

Leurs caract{\'e}ristiques sont les suivantes:
\begin{quote}
{\sl Synoptique~:}
\begin{quote}
\begin{verbatim}
#include <sys/types.h>
#include <sys/socket.h>

int   getsockopt (sk, level, optname, optval, optlen);
int   setsockopt (sk, level, optname, optval, optlen);
int   sk, level, optname, *optlen;
char *optval;
\end{verbatim}
\end{quote}

{\sl Avec~:}
\begin{quote}
\begin{tabular}{lp{8cm}}
{\tt sk}		& Descripteur sur la socket	\\
{\tt level}	& Niveau du protocole: dans ce cas, la valeur {\`a} pr{\'e}ciser est {\tt SOL\_SOCK}	\\
{\tt optname}	& Nom de l'option	\\
{\tt optval}	& Pointeur sur les informations {\`a} transmettre ou re\c{c}ues	\\
{\tt optlen}	& Volume en octet des informations transmises ou re\c{c}ues.	\\
\end{tabular}
\end{quote}

{\sl Codes de retour~:}
\begin{quote}
\begin{tabular}{lp{8cm}}
	Ok	& Renvoie le descripteur de socket	\\
	Erreur	& Renvoie la valeur \verb=-1=, la variable syst{\`e}me {\tt errno}
		  indiquera avec plus de pr{\'e}cision les causes de l'erreur
\end{tabular}
\end{quote}
\end{quote}

%%%%%%%%%%%
\subsection{\label{advnet-sockopt-stream}Options pour les sockets en mode connect{\'e}}

%%%%%%%%%%%
\subsubsection{Option {\tt SO\_KEEPALIVE}}

Lorsque l'option {\tt SO\_KEEPALIVE} est positionn{\'e}e, au niveau de la
couche {\sl transport}, le syst{\`e}me {\'e}met de fa\c{c}on p{\'e}riodique\footnote{L'intervalle de temps
entre deux �missions est fixe et peut avoir une dur�e entre 45 secondes et 6 minutes.}
depuis le port de communication local vers le port distant, en attendant un accus{\'e} de r{\'e}ception.
Sachant que ceci se passe directement au niveau de la couche {\sl transport}, il n'est pas
n{\'e}cessaire de rajouter du code {\`a} votre application. Cette option permet donc de contr{\^o}ler
que la connexion entre les deux processus est toujours active. Si, par hasard, le
port de communication distant n'est plus joignable, c'est {\`a} dire qu'il
ne r{\'e}pond plus aux messages p{\'e}riodiques qui lui ont {\'e}t{\'e} envoy{\'e}s, la
connexion est consid{\'e}r{\'e}e comme {\'e}tant coup{\'e}e. Dans ce cas~:
\begin{itemize}
	\item	le processus recevra le signal {\tt SIGPIPE} {\`a} la premi{\`e}re
		tentative d'{\'e}criture,
	\item	la premi{\`e}re tentative de lecture renverra une fin de fichier.
\end{itemize}

Cette option n'est disponible que pour les sockets en mode connect{\'e}
utilisant le protocole IP.

%%%%%%%%%%%
\subsubsection{Option {\tt SO\_DONROUTE}}

L'option {\tt SO\_DONROUTE} positionne une option dans l'ent{\^e}te IP
du paquet comme {\'e}tant {\it non routable}, c'est-{\`a}-dire que le paquet
ne peut {\^e}tre adress{\'e} qu'{\`a} une machine se trouvant dans le m{\^e}me r{\'e}seau
IP (cf. section \ref{protip-route}).

Par cons{\'e}quent, sachant que cette option ne s'applique qu'aux ports
de communication en mode connect{\'e}, un canal cr{\'e}{\'e} avec cet option 
sur un serveur, ne pourra satisfaire les requ{\^e}tes que des clients
dont l'adresse IP correspond au m{\^e}me num{\'e}ro de r{\'e}seau. De m{\^e}me, si un
client cr{\'e}e un canal avec cette option, il ne pourra {\'e}mettre des demandes
de connexions qu'{\`a} un serveur se trouvant dans le m{\^e}me r{\'e}seau IP.

Les routeurs ou {\sl gateways}\footnote{Les routeurs et les {\sl gateways}
sont des {\'e}l{\'e}ments actifs du r{\'e}seau permettant d'acheminer les datagrammes
IP d'un r{\'e}seau {\`a} l'autre. Les routeurs maintiennent des
tables de routages entre les diff{\'e}rents r{\'e}seaux IP disponibles et se chargent
d'acheminer les datagrammes du r{\'e}seau de d{\'e}part jusqu'au r{\'e}seau d'arriv{\'e}e.
Les {\sl gateways} sont des machines en g{\'e}n{\'e}ral sous {\Unix} poss{\'e}dant
deux interfaces r{\'e}seau, chacune appartenant {\`a} un r{\'e}seau IP distinct. Elles
assurent alors une fonction de {\sl passerelle} entre les deux r{\'e}seaux IP~:
les datagrammes {\`a} destination de l'un sont r{\'e}{\'e}mis sur l'autre, et
vice-versa.} ne prendront pas en compte ce type de paquet.

%%%%%%%%%%%
\subsubsection{Options {\tt SO\_SNDBUF} et {\tt SO\_RCVBUF}}

Les options {\tt SO\_SNDBUF} et {\tt SO\_RCVBUF} permettent de changer
la taille des buffers de communication au niveau logiciel de la socket
respectivement en {\'e}mission et en r{\'e}ception. Il est possible d'accro{\^\i}tre
ces buffers {\`a} tout moment du programme. Par contre, la diminution de
la taille des buffers ne peut se faire qu'au moment de l'{\'e}tablissement
de la connexion. La taille maximum des buffers est de 65 535 octets.

Ces options sont disponibles pour les sockets sur IP et celles de
type {\Unix} (cf. chapitre \ref{unstream}).

%%%%%%%%%%%
\subsection{\label{advnet-sockopt-other}Autres options}

%%%%%%%%%%%
\subsubsection{Option {\tt SO\_REUSEADDR}}

Nous avons vu que {\tt bind()} interdisait d'avoir deux ports
de communications avec le m{\^e}me nom, c'est-{\`a}-dire avec le couple
({\sl adresse-ip},{\sl num{\'e}ro-port}) identique pour deux sockets
diff{\'e}rentes.

Maintenant supposons qu'une connexion soit {\'e}tablie sur le port <<~3030~>>
avec la machine dont l'adresse IP est <<~192.0.0.103~>>. Supposons que
le serveur veuille cr{\'e}er une file d'attente sur ce port de communication
<<~3030~>> sans aucune restriction d'adresse, c'est-{\`a}-dire avec
l'option {\tt INADDR\_ANY}. Dans ce cas, l'appel {\`a} {\tt bind()} va
{\'e}chouer car, en utilisant l'adresse <<~{\tt *.3030}~>>, {\tt bind()} va
rentrer en conflit avec le port dont l'adresse est <<~{\tt 192.0.0.103.3030}~>>.

Contrairement {\`a} ce qe l'on pourrait penser, cette situation n'a rien 
d'exceptionnel pour un programme offrant un service sur le r{\'e}seau.
Pour cela il suffit qu'un processus serveur suive la s{\'e}quence
suivante~:
\begin{order}
	\item	il a effectu{\'e} un appel {\`a} {\tt bind()} et {\tt listen()} sur
		une adresse contenant une adresse g{\'e}n{\'e}rique ({\sl wildcard}
		{\tt INADDR\_ANY}),
	\item	il a d{\'e}ja {\'e}tablit une connexion avec un client,
	\item	il s'est interrompu de fa\c{c}on anormale sans avoir lib{\'e}rer les ressources,
	\item	il est red{\'e}marr{\'e} par l'op{\'e}rateur,
	\item	il doit donc {\`a} nouveau effectu{\'e} un appel {\`a} {\tt bind()} et �
		{\tt listen()} sur une adresse contenant une adresse
		g{\'e}n{\'e}rique ({\sl wildcard} {\tt INADDR\_ANY}), appels qui vont {\'e}chouer.
\end{order}

Lorsque l'option {\tt SO\_REUSEADDR} est positionn{\'e}e, {\tt bind()}
va ignorer le fait qu'une connexion en cours, va satisfaire l'adresse
g{\'e}n{\'e}rique pr{\'e}cis{\'e}e en argument. Dans ce cas, lorsque le serveur
va red{\'e}marrer, {\tt bind()} se terminera normalement et nous pourrons
continuer l'ex{\'e}cution du serveur normalement.

\begin{remarque}
Cette option n'est applicable que pour les sockets s'appuyant sur IP et
fonctionnant en mode connect{\'e} (famille d'adresses {\tt AF\_INET}, famille
de protocoles {\tt SOCK\_STREAM}), elle ne fonctionne pas avec les <<~{\Unix} Stream Sockets~>>.
\end{remarque}

\begin{remarque}
Une fois que cette option est positionn{\'e}e, il n'est plus possible de l'annuler.
\end{remarque}

%%%%%%%%%%%
\subsubsection{Options {\tt SO\_LINGER} et {\tt SO\_DONTLINGER}}

Les options {\tt SO\_LINGER} et {\tt SO\_DONTLINGER} affectent le
fonctionnement de l'appel syst{\`e}me {\tt close()}.

Lorsque l'option {\tt SO\_LINGER} est positionn{\'e}e avec, comme option,
une valeur non nulle de temporisation ({\sl timeout})\footnote{cf.
section \ref{advnet-sockopt-syscall} sur l'appel syst{\`e}me
{\tt setsockopt()}}, {\tt close()} bloquera l'ex{\'e}cution du programme
tant que des donn{\'e}es sont pr�sentes dans les {\sl buffers} syst{\`e}mes pour
{\^e}tre {\'e}mises, ou bien tant que le protocole TCP n'aura pas perdu
la connexion avec la machine distante (coupure de la connexion TCP
au bout de 6 minutes de non-r{\'e}ponse de la machine distante). Ce type
de fermeture de connexion peut {\^e}tre vue comme une {\sl fermeture en
douceur} du canal de communication ({\sl graceful close}).

\begin{remarque}
Cette temporisation doit {\^e}tre consid{\'e}r{\'e}e comme une valeur bool\-{\'e}enne.
Elle ne va pas changer la valeur du d{\'e}lai maximal du protocole TCP.
Si cette valeur est non nulle, alors le programme bloquera jusqu'{\`a} ce
que les donn{\'e}es restant dans les {\sl buffers} syst{\`e}mes pour
{\^e}tre {\'e}mises, ou bien jusqu'{\`a} ce que le protocole TCP perde
la connexion avec la machine distante.
\end{remarque}

Dans le cas o{\`u} cette valeur de temporisation est nulle, le canal est ferm{\'e} sans
attendre. Ce type de fermeture de connexion peut {\^e}tre
vue comme une {\sl fermeture s{\`e}che} du canal de communication ({\sl hard
close}).

L'option {\tt SO\_DONTLINGER} correspond au fonctionnement par d{\'e}faut,
c'est-{\`a}-dire que la fermeture du canal correspond {\`a} une fermeture en
douceur tout en faisant en sorte que {\tt close()} rende imm{\'e}iatement la
main. Par cons{\'e}quent, lorsque le programme fait appel {\`a} {\tt close()},
le syst{\`e}me lui rend la main imm{\'e}diatement, mais ex{\'e}cute en parall{\`e}le
tout ce qui correspond {\`a} une fermeture en douceur du canal de communication.

Le tableau \ref{advnet-linger-tab} r{\'e}sume les diff{\'e}rents comportements
obtenus en fonction des options {\tt SO\_LINGER} et {\tt SO\_DONTLINGER}.

\begin{table}[hbtp]
\centering
\begin{tabular}{|p{2.5cm}|c|c|c|}
	\hline
	Options & \parbox{3cm}{Temporisation}
			& \parbox{3cm}{Fermeture en douceur}
			& \parbox{3cm}{Fermeture s{\`e}che} \\
	\hline \hline
	{\tt SO\_DONTLINGER}	& {\sl N$/$A}		& Oui		&	\\
	\hline
	{\tt SO\_LINGER}		& Nul			& 		& Oui	\\
					& Non Nul		& Oui		&	\\
	\hline
\end{tabular}\\[3ex]
\begin{tabular}{|p{2.5cm}|c|c|c|}
	\hline
	Options & \parbox{3cm}{Temporisation}
			& \parbox{3cm}{Blocage de {\tt close()}}
			& \parbox{3cm}{
				{\tt close()} rend la main imm{\'e}diatement}	\\
	\hline \hline
	{\tt SO\_DONTLINGER}	& {\sl N$/$A}		&		& Oui \\
	\hline
	{\tt SO\_LINGER}		& Nul			&		& Oui \\
					& Non Nul		& Oui		& 	  \\
	\hline
\end{tabular}
\caption{\label{advnet-linger-tab}Diff{\'e}rents comportements obtenus en fonction des options
{\tt SO\_LINGER} et {\tt SO\_DONTLINGER}}
\end{table}

{\bf Exemple}
\begin{quote}
\begin{verbatim}
setsockopt (sk, SOL_SOCKET, SO_LINGER, "1", 1)
\end{verbatim}
\end{quote}

%%%%%%%%%%%%%%%%%%%%%%%%%%%%%%%%%%%%%%%%%%%%%%%%%%%%%%%%%%%%%%%%%%%%%
\section{Appel syst{\`e}me {\tt shutdown()}}

\subsection{Introduction, Description}

L'appel syst{\`e}me {\tt shutdown()} peut {\^e}tre utilis{\'e} {\`a} la place
de {\tt close()}. Alors que {\tt close()} rend l'identifiant du port de communication invalide (retour de l'appel syst{\`e}me
{\tt socket()}), {\tt shutdown()} peut {\^e}tre utilis{\'e} pour fermer partiellement ou totalement ce canal.

Ses caract{\'e}ristiques sont les suivantes~:
\begin{quote}
{\sl Synoptique~:}
\begin{quote}
\begin{verbatim}
int   shutdown (sk, how);
int   sk, how;
\end{verbatim}
\end{quote}

{\sl Avec~:}
\begin{quote}
\begin{tabular}{ll}
{\tt sk}		& Descripteur sur la socket.			\\
{\tt how}	& Description de la fermeture du canal.	\\
\end{tabular}
\end{quote}

{\sl Codes de retour~:}
\begin{quote}
\begin{tabular}{lp{8cm}}
	Ok	& Renvoie le descripteur de socket	\\
	Erreur	& Renvoie la valeur \verb=-1=, la variable syst{\`e}me {\tt errno}
		  indiquera avec plus de pr{\'e}cision les causes de l'erreur.
\end{tabular}
\end{quote}
\end{quote}

Les valeurs possibles pour l'argument {\tt how}, pr{\'e}cisant le mode de fermeture, sont d{\'e}crites dans le tableau
\ref{advnet-shutdown-values}.

\begin{table}[hbtp]
\centering
\begin{tabular}{|l|p{10cm}|}
	\hline
	Valeur	&	Description \\
	\hline
	{\tt 0}	&
	Il est possible d'{\'e}mettre toujours des donn{\'e}es sur le canal, mais
	il ne sera plus possible d'en recevoir. Que ce soit les donn{\'e}es en
	attente d'{\'e}mission (c'est-{\`a}-dire encore dans les {\sl buffers}
	syst{\`e}me) ou pour les futures donn{\'e}es {\`a} {\'e}mettre, les appels syst{\`e}mes
	{\tt send()}, {\tt sendto()} ou {\tt write()} ne g{\'e}n{\`e}reront pas
	d'erreurs. Par contre, toute op{\'e}ration de lecture sur ce canal
	renverra une fin de fichier. \\
	\hline
	{\tt 1}	&
	Seules les donn{\'e}es en attente d'{\'e}mission (c'est-{\`a}-dire encore dans
	les {\sl buffers} syst{\`e}me) pourront {\^e}tre {\'e}mises. Toute op{\'e}ration
	d'{\'e}mission de donn{\'e}es renverra l'erreur {\tt EPIPE} dans la variable
	{\tt errno}. De m{\^e}me, le noyau enverra le signal {\tt SIGPIPE} au
	processus. Par contre, il sera toujours possible de recevoir des donn{\'e}es.\\
	\hline
	{\tt 2}	&
	Cette option combine les deux cas pr{\'e}c{\'e}dents. Seules les donn{\'e}es
	en attente d'{\'e}mission seront trait{\'e}es avant la fermeture. Vous ne
	pourrez ni {\'e}mettre ni recevoir de donn{\'e}es.
	\\
	\hline
\end{tabular}
\caption{\label{advnet-shutdown-values}Options de comportement pour l'appel syst{\`e}me {\tt shutdown()}}
\end{table}

Par cons{\'e}quent,
\begin{itemize}
	\item	dans le cas d'un canal de communication fonctionnant en
		mode non-connect{\'e} ({\sl datagram socket}), {\tt shutdown}
		le marque comme ne pouvant plus utiliser les appels
		{\tt sendto()} ou {\tt recvfrom()}, en fonction de la valeur
		de l'argument {\tt how}. Une fois que ce type de canal ne
		peut ni {\'e}mettre ni recevoir de donn{\'e}es, il est ferm{\'e}
		compl{\`e}tement.
	\item	dans le cas d'un canal de communication fonctionnant en mode
		connect{\'e}, lorsque {\tt how} vaut <<~1~>> ou <<~2~>>, la connexion
		se termine <<~{\sl en douceur}\footnote{Le terme anglo-saxon
		utilis{\'e} est <<~{\sl graceful disconnect~}>>.}~>>. Le processus
		de d{\'e}connexion est termin{\'e} lorsque les deux parties en
		pr{\'e}sence ont fait appel {\`a} {\tt shutdown()}. {\tt shutdown}
		ne suspend pas l'ex{\'e}cution du programme.
\end{itemize}

%%%%%%%%%%%%%%%%%%%%%%%%%%%%%%%%%%%%%%%%%%%%%%%%%%%%%%%%%%%%%%%%%%%%%
\subsection{Application~: Fermeture d'un canal de communication}

La description qui va suivre, montre comment il est possible d'{\'e}tablir 
un protocole entre le client et le serveur afin de fermer le canal de communication.

Le tableau \ref{advnet-closeconn} d{\'e}crit un des processus qu'il est possible de suivre.

Tout d'abord, l'un des processus doit {\^e}tre {\`a} l'initiative de la s{\'e}quence {\`a}
suivre pour la fermeture du canal. Celui-ci va envoyer un {\it drapeau}
{\`a} tous les autres processus avec lesquels il veut arr{\^e}ter de
communiquer. Ces derniers vont donc recevoir cette informations, ils
continueront {\`a} {\'e}mettre des donn{\'e}es si cela est n{\'e}cessaire. Le processus,
{\`a} l'initiative de la d{\'e}connexion va donc continuer {\`a} recevoir des
donn{\'e}es. Lorsque tous les processus ayant re\c{c}us la demande de
d{\'e}connexion auront termin{\'e} d'{\'e}mettre toutes leurs donn{\'e}es, ils r{\'e}{\'e}mettront
le {\it drapeau} vers processus initiateur et fermeront leur canal de
communication. Le processus initiateur va donc recevoir, en retour, le
{\it drapeau} dont il a {\'e}t{\'e} {\`a} l'origine. Il peut alors fermer son
ou ses canaux de communication.


\begin{table}[hbtp]
\begin{tabular}{|p{7cm}|p{7cm}|}
	\hline
	envoie message <<~{\sl d{\'e}connexion}~>>	&	r{\'e}ception de donn{\'e}es					\\
	tant que des donn{\'e}es sont {\`a} recevoir	&	tant que des donn{\'e}es doivent {\^e}tre {\'e}mises		\\
	\hspace{1cm}r{\'e}ception des donn{\'e}es	&	\hspace{1cm}{\'e}mission des donn{\'e}es			\\
	fermeture du canal				&	si le message <<~{\sl d{\'e}connexion}~>> est re\c{c}u	\\
							&	\hspace{1cm}fermeture du canal				\\
	\hline
\end{tabular}
\caption{\label{advnet-closeconn}S{\'e}quence de d{\'e}connexion entre le client et le serveur}
\end{table}