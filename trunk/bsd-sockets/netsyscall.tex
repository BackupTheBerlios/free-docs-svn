%%%%%%%%%%%%%%%%%%%%%%%%%%%%%%%%%%%%%%%%%%%%%%%%%%%%%%%%%%%%%%%%%%%%%
%
% Les appels syst{\`e}mes associ{\'e}s aux services r{\'e}seau
%
\chapter{\label{nsyscall}Les appels syst{\`e}mes associ{\'e}s aux services r{\'e}seau}

Il existe un ensemble de fonctions permettant de faciliter la
programmation avec les {\sl BSD Sockets}. Tous les appels qui seront
d{\'e}taill{\'e}s dans ce chapitre seront ceux disponibles sous {\Unix}. Toutefois,
ils pourront {\^e}tre aussi accessibles sur d'autres
syst{\`e}mes. De fa\c{c}on g{\'e}n{\'e}rale, toutes les r{\'e}f{\'e}rences qui seront faites
dans ce chapitre, se baseront sur le syst{\`e}me {\Unix}.

Nous verrons les points suivants:
\begin{itemize}
	\item	comment acc{\'e}der aux informations sur les canaux de
		communication disponibles au niveau d'un processus,
	\item	les m{\'e}thodes pour obtenir les informations dans les
		fichiers de configurations, ou bien les diff{\'e}rents services
		d'annuaires, disponibles sur le syst{\`e}me\footnote{Les syst�mes d'annuaires ici repr�sentent,
		pour la partie qui nous interesse, la liste de correspondance entre nom de machine et adresse IP
		({\tt /etc/hosts}), nom et num{\'e}ro de r{\'e}seau IP ({\tt /etc/networks}),
		nom et num{\'e}ro de protocoles ({\tt /etc/protocols}), nom et
		num{\'e}ro de service r{\'e}seau ({\tt /etc/services}).} ,
	\item	les fonctions permettant de convertir et de manipuler les adresses r{\'e}seau IP.
\end{itemize}

%%%%%%%%%%%%%%%%%%%%%%%%%%%%%%%%%%%%%%%%%%%%%%%%%%%%%%%%%%%%%%%%%%%%%
\section{Informations sur les canaux de communication}

Pour disposer des information sur des canaux de communication actifs, deux
appels syst{\`e}mes sont disponibles~:
\begin{itemize}
	\item	{\tt getsockname()} permet {\`a} un processus d'obtenir toutes les
			informations disponibles sur un port de communication,
	\item	{\tt getpeername()} permet {\`a} un processus d'obtenir les
informations
			disponibles sur le port de communication distant, c'est-{\`a}-dire
			la socket distante li{\'e}e {\`a} la socket locale.
\end{itemize}

Ces deux appels syst{\`e}mes sont souvent utilis{\'e}s lors de phases de
mise au point, afin d'{\'e}tablir des diagnostics sur le fonctionnement des communications.

Leurs cract{\'e}ristiques sont les suivantes~:
\begin{quote}
{\sl Synoptique~:}
\begin{quote}
\begin{verbatim}
int getsockname (sk, &addr, &addrlen);
int getpeername (sk, &addr, &addrlen);

int             sk,
                addrlen;
struct sockaddr addr;
\end{verbatim}
\end{quote}

{\sl Avec~:}
\begin{quote}
\begin{tabular}{ll}
{\tt s}		& Descripteur de la socket (cr{\'e}{\'e} lors de l'appel {\`a} {\tt socket()}).	\\
{\tt addr}	& Structure de d{\'e}finition des caract{\'e}ristiques de la socket.		\\
{\tt addrlen}	& Taille en octet de la structure de d{\'e}finition.				\\
\end{tabular}
\end{quote}

{\sl Codes de retour~:}
\begin{quote}
\begin{tabular}{lp{8cm}}
	Ok	& Renvoie le descripteur de socket	\\
	Erreur	& Renvoie la valeur \verb=-1=, la variable syst{\`e}me {\tt errno}
			  indiquera avec plus de pr{\'e}cision les causes de l'erreur
\end{tabular}
\end{quote}
\end{quote}

Comme pour les appels syst{\`e}mes {\tt connect()}, {\tt bind()} et {\tt accept()}
(cf. sections \ref{instream-syscall-bind}, \ref{instream-syscall-connect} et \ref{instream-syscall-accept})
la structure d{\'e}crivant les caract{\'e}ristiques de la socket d{\'e}pend de la famille de protocoles utilis{\'e}e.
Dans le cas qui nous interresse, nous utilisons le protocole IP, par cons{\'e}quent, le type de la
structure qui sera renvoy{\'e}e sera {\tt struct sockaddr\_in}.
L'argument {\tt addrlen} contiendra la taille de la structure ainsi initialis{\'e}e
\footnote{Le contenu des arguments {\tt addr} et {\tt addrlen} sera modifi{\'e}
par les deux appels syst{\`e}mes {\tt getsockname()} et {\tt getpeername()}.
L'allocation m{\'e}moire devra donc {\^e}tre d{\'e}j{\`a} faite avant tout appel.}.

%%%%%%%%%%%%%%%%%%%%%%%%%%%%%%%%%%%%%%%%%%%%%%%%%%%%%%%%%%%%%%%%%%%%%
\section{\label{netsyscall-infohost}Informations sur la liste des machines du r{\'e}seau}

%%%%%%%%%%%%%%%%%%%%%%%%%%%%%%%%%%%%%%%%%%%%%%%%%%%%%%%%%%%%%%%%%%%%%
\subsection{Rappel sur le fichier {\tt $/$etc$/$hosts}}

{\`A} la section \ref{protip-etc-hosts}, nous avons vu la structure du
fichier {\tt /etc/hosts}. Ce fichier permet de faire la liaison entre
l'adresse IP (adresse r{\'e}seau) d'une machine et un nom. Il existe d'autres
m{\'e}canismes d'annuaires sur les syst{\`e}mes {\Unix} assurant cette fonctionnalit{\'e}~:
\begin{itemize}
	\item	le service NIS\footnote{Network Information Service} de Sun Microsystems,
	\item	le service DNS\footnote{Domain Name Server} disponible sur d'autres
		syst{\`e}mes d'exploitation. C'est ce service qui est utilis{\'e} sur le
		r{\'e}seau Internet.
\end{itemize}

Quelque soit le syst{\`e}me d'annuaire utilis{\'e}, on retrouvera, pour une adresse
r{\'e}seau donn{\'e}e, les informations suivantes~:
\begin{description}
	\item[l'adresse r{\'e}seau (adresse IP)]\mbox{}\\
		Cette adresse doit {\^e}tre unique sur l'ensemble du r{\'e}seau consid{\'e}r{\'e}.
	\item[le nom officiel de la machine]\mbox{}\\
		Ce nom n'est pas obligatoirement unique. Dans ce cas, seule la premi{\`e}re
		entr{\'e}e trouv{\'e}e sera prise en compte.
	\item[les noms secondaires]\mbox{}\\
		Ces noms correspondent {\`a} la notion d'{\sl aliases} pour une machine
		donn{\'e}e. Elle pourra donc {\^e}tre appel{\'e}e, soit par son nom officiel, soit par l'un
		des noms secondaires. Par exemple, un serveur Web peut avoir son 
		nom officiel et le nom commun{\'e}ment utilis{\'e} sur Internet~:
		<<~{\tt www}~>>.
\end{description}

{\bf Exemple~:}
\begin{quote}
\begin{verbatim}
192.0.0.1    bart  ftp
192.0.0.2    homer www
\end{verbatim}
\end{quote}

%%%%%%%%%%%%%%%%%%%%%%%%%%%%%%%%%%%%%%%%%%%%%%%%%%%%%%%%%%%%%%%%%%%%%
\subsection{\label{nsyscall-sysc-hosts}Appels syst{\`e}mes et structures associ{\'e}s}

La biblioth�que de fonctions standards r{\'e}seau contient un certain nombre de
fonctions permettants d'obtenir les informations contenues dans le fichier
{\tt /etc/hosts} (ou {\'e}quivalent). On aura~:
\begin{description}
	\item[{\tt gethostent()}]\mbox{}\\
		Les appels successifs {\`a} {\tt gethostent()} permettent d'obtenir
		les entr{\'e}es disponible dans la table de traduction <<~adresse IP $/$ nom de machine~>>.
	\item[{\tt gethostbyname()}]\mbox{}\\
		Permet d'effectuer une recherche et d'obtenir les informations
		de la premi{\`e}re entr{\'e}e en fonction d'un nom officiel pour une machine.
	\item[{\tt gethostbyaddr()}]\mbox{}\\
		A la m{\^e}me fonctionnalit{\'e} que la fonction {\tt gethostbyname()}, mais
		la recherche s'effectue en fonction d'une adresse r{\'e}seau.
\end{description}

Comme nous l'avons vu, la fonction {\tt gethostent()} va renvoyer les
informations sur l'entr{\'e}e courante de l'annuaire {\`a} chaque fois qu'elle
sera appel{\'e}e. Avant de rendre la main, elle pr{\'e}parera le contexte pour que le
prochain appel donne les informations sur l'entr{\'e}e suivante. Pour cela, il 
faut donc cr{\'e}er un contexte d'ex{\'e}cution et aussi disposer d'une fonction y
mettant fin~:
\begin{description}
	\item[{\tt sethostent()}]\mbox{}\\
		Cette fonction initialise le contexte. Elle ouvre le fichier
		{\tt /etc/hosts} (ou tout autre syst{\`e}me d'annuaire) et positionne
		le pointeur de lecture au d{\'e}but. Le contexte est pr{\^e}t pour faire
		des appels successifs avec la fonction {\tt gethostent()}.
	\item[{\tt endhostent()}]\mbox{}\\
		Cette fonction ferme le contexte.
\end{description}

\begin{remarque}
{\tt gethostbyname()} et {\tt gethostbyaddr()} n'influent pas sur le
contexte d'ex{\'e}cution positionn{\'e} par {\tt sethostent()} et {\tt endhostent()}.
Seule la fonction {\tt gethostent()} l'utilisera.
\end{remarque}

Les fonctions {\tt gethostent()}, {\tt gethostbyname()} et {\tt gethostbyaddr()}
poss{\`e}dent les caract{\'e}ristiques suivantes~:
\begin{quote}
{\sl Synoptique~:}
\begin{quote}
\begin{verbatim}
hp =  gethostent ();
hp =  gethostbyname (name);
hp =  gethostbyaddr (sk, addr, addrlen, type);

struct hostent *hp;
char           *name;
addr_struct     addr;
int             addrlen, type;
\end{verbatim}
\end{quote}

{\sl Avec~:}
\begin{quote}
\begin{tabular}{lp{8cm}}
{\tt hp}		&	Structure de type {\tt hostent} d{\'e}crivant les informations sur
			la machine correspondant {\`a} l'entr{\'e}e courante du fichier 
			{\tt /etc/hosts}.	\\
{\tt name}	&	Chaine de caract{\`e}res contenant le nom de la machine dont
			on d{\'e}sire obtenir les informations.\\
{\tt addr}	&	Structure d{\'e}crivant l'adresse recherch�e. \\
{\tt addrlen}	&	Taille en octet de la structure d{\'e}crivant l'adresse r{\'e}seau. \\
{\tt type}	&	Type d'adresse r{\'e}seau utilis{\'e}e (famille d'adresse).
\end{tabular}
\end{quote}

{\sl Codes de retour~:}
\begin{quote}
\begin{tabular}{lp{8cm}}
	Ok	& Renvoie un pointeur sur une structure de type {\tt hostent}.	\\
	Erreur	& Renvoie le pointeur \verb=NULL=.	\\
\end{tabular}
\end{quote}
\end{quote}

\begin{description}
	\item[{\tt gethostent()}]\mbox{}\\
		Cette fonction ne poss{\`e}de pas d'arguments. Elle fonctionne en collaboration
		avec les fonctions {\tt sethostent()} et {\tt endhostent()}. Elle
		retourne un pointeur sur une structure de type {\tt hostent}, structure
		contenant les informations sur une entr{\'e}e du fichier {\tt /etc/hosts}.
		Cette fonction est souvent utilis{\'e}e pour retourner les informations
		sur toutes les machines ou seulement une partie contenue dans {\tt /etc/hosts}.
	\item[{\tt gethostbyname()}]\mbox{}\\
		Cette fonction poss{\`e}de un argument~: le nom de la machine {\`a} rechercher.
		Elle retourne un pointeur sur une structure de type {\tt hostent}. Le
		nom pr{\'e}cis{\'e} en argument peut {\^e}tre le nom officiel de la machine ou bien
		l'un de ses {\sl aliases} (noms secondaires). Lorsque le syst{\`e}me
		supporte le {\sl multi-homing}, il est possible d'avoir correspondance entre un seul nom et plusieurs
		adresses r{\'e}seau. Dans ce cas, {\tt gethostbyname()} renverra la premi{\`e}re
		entr{\'e}e associ{\'e} au nom de machine pass{\'e} en argument qu'il trouvera.
	\item[{\tt gethostbyaddr()}]\mbox{}\\
		Cette fonction poss{\`e}de trois arguments,
		\begin{itemize}
			\item	{\tt addr}, une structure d{\'e}crivant l'adresse {\`a} rechercher.
				Le type de cette structure d{\'e}pendra de l'adresse r{\'e}seau
				utilis{\'e}e. Evidemment, on utilisera usuellement les adresses
				IP, donc une structure du type {\tt inaddr}. Il sera toutefois
				possible de faire une recherche sur des adresses AppleTalk,
				DECnet, SNA, etc.
			\item	{\tt addrlen} sp{\'e}cifie la taille en octet de la structure
				permettant de pr{\'e}ciser l'adresse r{\'e}seau recherch{\'e}e.
			\item	{\tt type} permet de pr{\'e}ciser quel est le type d'adresse
				r{\'e}seau sp{\'e}cifi{\'e} en argument. Pour
				cela, on utilisera les m{\^e}mes constantes que pour l'appel
				syst{\`e}me <<~{\tt socket()}~>>. Par cons{\'e}quent, pour les adresses
				IP, on sp{\'e}cifiera <<~{\tt AF\_INET}~>>.
		\end{itemize}
		Quelque soit la famille d'adresses utilis{\'e}e, une adresse r{\'e}seau ne pourra
		correspondre qu'{\`a} une et une seule machine. Il n'y aura donc aucun probl{\`e}me
		d'ambigu{\"\i}t{\'e}.
\end{description}

\begin{remarque}
Avec la notion de grappes de machines (cluster), une adresse peut r�f�rencer une groupement de machines
physiques. Dans ce cas, l'adresse r�f�rence la machine logique ainsi constitu�e. Une machine de la grappe
dispose donc de deux adresses~:
\begin{itemize}
	\item	celle la r�f�ren�ant en tant que tel,
	\item	celle la r�f�ren�ant en tant que membre du groupement de machines.
\end{itemize}

Sans aller chercher aussi loin, une machine peut disposer de plusieurs adresses tout simplement si
elle dispose de plusieurs cartes ou point d'acc�s au r�seau.
\end{remarque}

La structure {\tt hostent} est compos{\'e}e des champs suivants~:\\
\begin{tabular}{lp{8cm}}
	{\tt h\_name}		&	nom officiel de la machine (chaine de caract{\`e}res).\\[1.5ex]
	{\tt h\_aliases}		&	un tableau de chaines de caract{\`e}res contenant la liste
					des noms secondaires de la machine. Le dernier {\'e}l{\'e}ment
					du tableau correspond au pointeur \verb=NULL=.\\[1.5ex]
	{\tt h\_addrtype}	&	type de l'adresse r{\'e}seau renvoy{\'e}e. Pour les adresses
					IP, ce champ contient la constante <<~{\tt AF\_INET}~>>.\\[1.5ex]
	{\tt h\_length}		&	longueur, en octet, de l'adresse renvoy{\'e}e.\\[1.5ex]
	{\tt h\_addr\_list}	&	pointeur sur une zone m{\'e}moire contenant les adresses
					r{\'e}seau de la machine, le dernier �l�ment du tableau correspondra au
					pointeur NULL (\texttt{h\_addr\_list[\textit{last}] == NULL}).\\
\end{tabular}

%%%%%%%%%%%%%%%%%%%%%%%%%%%%%%%%%%%%%%%%%%%%%%%%%%%%%%%%%%%%%%%%%%%%%
\section{\label{nsyscall-infnet}Informations sur la liste des r{\'e}seaux}

%%%%%%%%%%%%%%%%%%%%%%%%%%%%%%%%%%%%%%%%%%%%%%%%%%%%%%%%%%%%%%%%%%%%%
\subsection{Rappel sur le fichier {\tt $/$etc$/$networks}}

{\`A} la section \ref{protip-etc-network}, nous avons vu la structure du
fichier {\tt /etc/networks}. Ce fichier permet de faire la liaison entre
un num{\'e}ro de r{\'e}seau IP et un nom. Il existe d'autres
m{\'e}canismes d'annuaires sur les syst{\`e}mes {\Unix} assurant cette
fonctionnalit{\'e}~: le service NIS de Sun Microsystems.

Quelque soit le syst{\`e}me d'annuaire utilis{\'e}, on retrouvera, pour un num{\'e}ro
de r{\'e}seau donn{\'e}, les informations suivantes~:
\begin{description}
	\item[le nom officiel du r{\'e}seau]\mbox{}\\
		Ce nom n'est pas obligatoirement unique. Dans ce cas, seule la premi{\`e}re
		entr{\'e}e trouv{\'e}e sera prise en compte.
	\item[le num{\'e}ro de r{\'e}seau]\mbox{}\\
		Ce num{\'e}ro doit {\^e}tre unique sur l'ensemble du r{\'e}seau consid{\'e}r{\'e}.
	\item[les noms secondaires]\mbox{}\\
		Ces noms correspondent {\`a} la notion d'{\sl aliases} pour un r{\'e}seau
		donn{\'e}, c'est-{\`a}-dire qu'un r{\'e}seau dispose de plusieurs noms.
\end{description}

{\bf Exemple~:}
\begin{quote}
\begin{verbatim}
192.155.2   mon-reseau
144.2       reseau-B
193.1.2	    autre-reseau
13          reseau-A
\end{verbatim}
\end{quote}

Pour plus de pr{\'e}cisions sur la notation {\`a} utiliser sur lesnum{\'e}ros de r{\'e}seaux
avec les adresses IP, reportez vous {\`a} la section \ref{protip-defIP}.

%%%%%%%%%%%%%%%%%%%%%%%%%%%%%%%%%%%%%%%%%%%%%%%%%%%%%%%%%%%%%%%%%%%%%
\subsection{Appels syst{\`e}mes et structures associ{\'e}s}

La biblioth�que de fonctions standards r{\'e}seau contient un certain nombre de
fonctions permettants d'obtenir les informations contenues dans le fichier
{\tt /etc/networks} (ou {\'e}quivalent). On aura~:
\begin{description}
	\item[{\tt getnetent()}]\mbox{}\\
		Les appels successifs {\`a} {\tt getnetent()} permettent d'obtenir
		les entr{\'e}es disponible dans la table de traduction <<~r{\'e}seau IP $/$ nom de r{\'e}seau~>>.
	\item[{\tt getnetbyname()}]\mbox{}\\
		Permet d'effectuer une recherche et d'obtenir les informations
		de la premi{\`e}re entr{\'e}e en fonction d'un nom officiel pour un r{\'e}seau.
	\item[{\tt getnetbyaddr()}]\mbox{}\\
		A la m{\^e}me fonctionnalit{\'e} que la fonction {\tt getnetbyname()}, mais
		la recherche s'effectue en fonction d'un num{\'e}ro de r{\'e}seau.
\end{description}

La fonction {\tt getnetent()} fonctionne de la m{\^e}me fa\c{c}on que la
fonction {\tt gethostent()}, c'est-{\`a}-dire qu'elle renverra les
informations sur l'entr{\'e}e courante de l'annuaire {\`a} chaque fois qu'elle
sera appel{\'e}e. Avant de rendre la main, elle pr{\'e}parera le contexte pour que le
prochain appel donne les informations sur l'entr{\'e}e suivante.Les fonctions permettant
de cr{\'e}er le contexte d'ex{\'e}cution sont~:
\begin{description}
	\item[{\tt setnetent()}]\mbox{}\\
		Cette fonction initialise le contexte. Elle ouvre le fichier
		{\tt /etc/networks} (ou tout autre syst{\`e}me d'annuaire) et positionne
		le pointeur de lecture au d{\'e}but. Le contexte est pr{\^e}t pour faire
		des appels successifs avec la fonction {\tt getnetent()}.
	\item[{\tt endnetent()}]\mbox{}\\
		Cette fonction ferme le contexte.
\end{description}

\begin{remarque}
De la m{\^e}me fa\c{c}on, {\tt getnetbyname()} et {\tt getnetbyaddr()} n'influent pas sur le
contexte d'ex{\'e}cution positionn{\'e} par {\tt setnetent()} et {\tt endnetent()}.
Seule la fonction {\tt getnetent()} l'utilisera.
\end{remarque}

Les fonctions {\tt getnetent()}, {\tt getnetbyname()} et {\tt getnetbyaddr()}
poss{\`e}dent les caract{\'e}ristiques suivantes~:
\begin{quote}
{\sl Synoptique~:}
\begin{quote}
\begin{verbatim}
np =  getnetent ();
np =  getnetnyname (name);
np =  getnetbyaddr (addr, addrlen, type);

struct netent *np;
char          *name;
addr_struct    addr;
int            addrlen, type;
\end{verbatim}
\end{quote}

{\sl Avec~:}
\begin{quote}
\begin{tabular}{lp{8cm}}
{\tt np}		&	Structure de type {\tt netent} d{\'e}crivant les informations sur
			le r{\'e}seau correspondant {\`a} l'entr{\'e}e courante du fichier 
			{\tt /etc/networks}.	\\
{\tt name}	&	Chaine de caract{\`e}res contenant le nom du r{\'e}seau dont
			on d{\'e}sire obtenir les informations.\\
{\tt addr}	&	Structure d{\'e}crivant le num{\'e}ro de r{\'e}seau {\`a} rechercher. \\
{\tt addrlen}	&	Taille en octet de la structure d{\'e}crivant l'adresse r{\'e}seau. \\
{\tt type}	&	Type d'adresse r{\'e}seau utilis{\'e}e (famille d'adresse).
\end{tabular}
\end{quote}

{\sl Codes de retour~:}
\begin{quote}
\begin{tabular}{lp{8cm}}
	Ok	& Renvoie un pointeur sur une structure de type {\tt netent}.	\\
	Erreur	& Renvoie le pointeur \verb=NULL=.	\\
\end{tabular}
\end{quote}
\end{quote}

\begin{description}
	\item[{\tt getnetent()}]\mbox{}\\
		Cette fonction ne poss{\`e}de pas d'arguments. Elle fonctionne en collaboration
		avec les fonctions {\tt setnetent()} et {\tt endnetent()}. Elle
		retourne un pointeur sur une structure de type {\tt netent}, structure
		contenant les informations sur une entr{\'e}e du fichier {\tt /etc/networks}.
		Cette fonction est souvent utilis{\'e}e pour retourner les informations
		sur tous les r{\'e}seaux ou seulement une partie contenue dans {\tt /etc/networks}.
	\item[{\tt getnetbyname()}]\mbox{}\\
		Cette fonction poss{\`e}de un argument~: le nom du r{\'e}seau {\`a} rechercher.
		Elle retourne un pointeur sur une structure de type {\tt netent}. Le
		nom pr{\'e}cis{\'e} en argument peut {\^e}tre le nom officiel du r{\'e}seau ou bien
		l'un de ses {\sl aliases} (ou noms secondaires). Comme pour le fichier
		{\tt /etc/hosts}, il est possible d'affecter un m{\^e}me nom {\`a} plusieurs
		r{\'e}seaux distincts~: c'est le {\sl multi-homing}. Dans ce cas,
		{\tt getnetbyname()} renverra la premi{\`e}re
		entr{\'e}e associ{\'e} au nom de r{\'e}seau pass{\'e} en argument qu'il trouvera.
	\item[{\tt getnetbyaddr()}]\mbox{}\\
		Cette fonction poss{\`e}de trois arguments,
		\begin{itemize}
			\item	{\tt addr}, une structure d{\'e}crivant le num{\'e}ro de r{\'e}seau
				{\`a} rechercher. Le type de cette structure d{\'e}pendra de
				l'adresse r{\'e}seau utilis{\'e}e. Evidemment, on utilisera usuellement
				les adresses IP, donc une structure du type {\tt inaddr}. Il
				sera toutefois possible de faire une recherche sur des adresses
				AppleTalk, DECnet, SNA, etc.
			\item	{\tt addrlen} sp{\'e}cifie la taille en octet de la structure
				permettant de pr{\'e}ciser le num{\'e}ro de r{\'e}seau recherch{\'e}.
			\item	{\tt type} permet de pr{\'e}ciser quel est le type d'adresse
				r{\'e}seau sp{\'e}cifi{\'e} en argument. Pour
				cela, on utilisera les m{\^e}mes constantes que pour l'appel
				syst{\`e}me <<~{\tt socket()}~>>. Par cons{\'e}quent, pour les adresses
				IP, on sp{\'e}cifiera <<~{\tt AF\_INET}~>>.
		\end{itemize}
\end{description}

La structure {\tt netent} est compos{\'e}e des champs suivants~:\\
\begin{tabular}{lp{8cm}}
	{\tt n\_name}		&	nom officiel du r{\'e}seau (chaine de caract{\`e}res).\\[1.5ex]
	{\tt n\_aliases}		&	un tableau de chaines de caract{\`e}res contenant la liste
					des noms secondaires du r{\'e}seau. Le dernier {\'e}l{\'e}ment
					du tableau correspond au pointeur \verb=NULL=.\\[1.5ex]
	{\tt n\_addrtype}	&	type de l'adresse r{\'e}seau renvoy{\'e}e. Pour les adresses
					IP, ce champ contient la constante <<~{\tt AF\_INET}~>>.\\[1.5ex]
	{\tt n\_addr}		&	pointeur sur une zone m{\'e}moire contenant le num{\'e}ro du
					r{\'e}seau.\\
\end{tabular}


%%%%%%%%%%%%%%%%%%%%%%%%%%%%%%%%%%%%%%%%%%%%%%%%%%%%%%%%%%%%%%%%%%%%%
\section{\label{nsyscall-infprot}Informations sur la liste des protocoles disponibles}

%%%%%%%%%%%%%%%%%%%%%%%%%%%%%%%%%%%%%%%%%%%%%%%%%%%%%%%%%%%%%%%%%%%%%
\subsection{Rappel sur le fichier {\tt $/$etc$/$protocols}}

{\`A} la section \ref{protip-etc-protocols}, nous avons vu la structure du
fichier {\tt /etc/protocols}. Ce fichier permet de faire la liaison entre
un num{\'e}ro de protocole r{\'e}seau et un nom. Il existe d'autres
m{\'e}canismes d'annuaires sur les syst{\`e}mes {\Unix} assurant cette
fonctionnalit{\'e}~: le service NIS de Sun Microsystems.

Quelque soit le syst{\`e}me d'annuaire utilis{\'e}, on retrouvera, pour un num{\'e}ro
de r{\'e}seau donn{\'e}, les informations suivantes~:
\begin{itemize}
	\item	le nom officiel du protocole,
	\item	le num{\'e}ro de protocole,
	\item	les {\'e}ventuels noms secondaires.
\end{itemize}

{\bf Exemple~:}
\begin{quote}
\begin{verbatim}
ip      0   IP
icmp    1   ICMP
igmp    2   IGMP
ggp     3   GGP
tcp     6   TCP
udp     17  UDP
\end{verbatim}
\end{quote}

%%%%%%%%%%%%%%%%%%%%%%%%%%%%%%%%%%%%%%%%%%%%%%%%%%%%%%%%%%%%%%%%%%%%%
\subsection{Appels syst{\`e}mes et structures associ{\'e}s}

La biblioth�que de fonctions standards r{\'e}seau contient un certain nombre de
fonctions permettants d'obtenir les informations contenues dans le fichier
{\tt /etc/protocols} (ou {\'e}quivalent). On aura~:
\begin{description}
	\item[{\tt getprotoent()}]\mbox{}\\
		Les appels successifs {\`a} {\tt getprotoent()} permettent d'obtenir
		les entr{\'e}es disponible dans la table de traduction <<~num{\'e}ro de
		protocole $/$ nom de r{\'e}seau~>>.
	\item[{\tt getprotobyname()}]\mbox{}\\
		Permet d'effectuer une recherche et d'obtenir les informations
		de la premi{\`e}re entr{\'e}e en fonction d'un nom officiel pour un protocole.
	\item[{\tt getprotobynumber()}]\mbox{}\\
		A la m{\^e}me fonctionnalit{\'e} que la fonction {\tt getprotobyname()}, mais
		la recherche s'effectue en fonction d'un num{\'e}ro de protocole.
\end{description}

La fonction {\tt getprotoent()} fonctionne de la m{\^e}me fa\c{c}on que les
fonctions {\tt gethostent()} et {\tt getnetent()}, c'est-{\`a}-dire qu'elle renverra
les informations sur l'entr{\'e}e courante de l'annuaire {\`a} chaque fois qu'elle
sera appel{\'e}e. Avant de rendre la main, elle pr{\'e}parera le contexte pour que le
prochain appel donne les informations sur l'entr{\'e}e suivante.Les fonctions permettant
de cr{\'e}er le contexte d'ex{\'e}cution sont~:
\begin{description}
	\item[{\tt setprotoent()}]\mbox{}\\
		Cette fonction initialise le contexte. Elle ouvre le fichier
		{\tt /etc/protocols} (ou tout autre syst{\`e}me d'annuaire) et positionne
		le pointeur de lecture au d{\'e}but. Le contexte est pr{\^e}t pour faire
		des appels successifs avec la fonction {\tt getprotoent()}.
	\item[{\tt endprotoent()}]\mbox{}\\
		Cette fonction ferme le contexte.
\end{description}

\begin{remarque}
	De la m{\^e}me fa\c{c}on, {\tt getprotobyname()} et {\tt getprotobynumber()}
	n'influent pas sur le contexte d'ex{\'e}cution positionn{\'e} par {\tt 	setprotoent()} et {\tt endprotoent()}.
	Seule la fonction {\tt getprotoent()} l'utilisera.
\end{remarque}

Les fonctions {\tt getprotoent()}, {\tt getprotobyname()} et {\tt getprotobynumber()}
poss{\`e}dent les caract{\'e}ristiques suivantes~:
\begin{quote}
{\sl Synoptique~:}
\begin{quote}
\begin{verbatim}
pp =  getprotoent ();
pp =  getprotonyname (name);
pp =  getprotobyaddr (number);

struct protoent *pp;
char            *name;
int              number;
\end{verbatim}
\end{quote}

{\sl Avec~:}
\begin{quote}
\begin{tabular}{lp{8cm}}
{\tt pp}		&	Structure de type {\tt protoent} d{\'e}crivant les informations sur
			le r{\'e}seau correspondant {\`a} l'entr{\'e}e courante du fichier 
			{\tt /etc/protocols}.	\\
{\tt name}	&	Chaine de caract{\`e}res contenant le nom du protocole dont
			on d{\'e}sire obtenir les informations.\\
{\tt number}	& 	Entier contenant le num{\'e}ro du protocole dont
			on d{\'e}sire obtenir les informations.\\
\end{tabular}
\end{quote}

{\sl Codes de retour~:}
\begin{quote}
\begin{tabular}{lp{8cm}}
	Ok	& Renvoie un pointeur sur une structure de type {\tt protoent}.	\\
	Erreur	& Renvoie le pointeur \verb=NULL=.\\
\end{tabular}
\end{quote}
\end{quote}

\begin{description}
	\item[{\tt getprotoent()}]\mbox{}\\
		Cette fonction ne poss{\`e}de pas d'arguments. Elle fonctionne en collaboration
		avec les fonctions {\tt setprotoent()} et {\tt endprotoent()}. Elle retourne un pointeur sur une
		structure de type {\tt protoent}, structure
		contenant les informations sur une entr{\'e}e du fichier {\tt /etc/protocols}.
	\item[{\tt getprotobyname()}]\mbox{}\\
		Cette fonction poss{\`e}de un argument~: le nom du protocole {\`a} rechercher.
		Elle retourne un pointeur sur une structure de type {\tt protoent}.
	\item[{\tt getprotobynumber()}]\mbox{}\\
		Cette fonction poss{\`e}de un argument~: le num{\'e}ro du protocole {\`a} rechercher.
		Elle retourne un pointeur sur une structure de type {\tt protoent}.
\end{description}

La structure {\tt protoent} est compos{\'e}e des champs suivants~:\\
\begin{tabular}{lp{8cm}}
	{\tt p\_name}		&	nom officiel du r{\'e}seau (chaine de caract{\`e}res).\\[1.5ex]
	{\tt p\_aliases}		&	un tableau de chaines de caract{\`e}res contenant la liste
					des noms secondaires du protocole. Le dernier {\'e}l{\'e}ment
					du tableau correspond au pointeur \verb=NULL=.\\[1.5ex]
	{\tt n\_proto}		&	le num{\'e}ro du protocole.\\
\end{tabular}

%%%%%%%%%%%%%%%%%%%%%%%%%%%%%%%%%%%%%%%%%%%%%%%%%%%%%%%%%%%%%%%%%%%%%
\section{\label{nsyscall-infosvc}Informations sur la liste des services r{\'e}seau disponibles}

%%%%%%%%%%%%%%%%%%%%%%%%%%%%%%%%%%%%%%%%%%%%%%%%%%%%%%%%%%%%%%%%%%%%%
\subsection{Rappel sur le fichier {\tt $/$etc$/$services}}

{\`A} la section \ref{protip-etc-services}, nous avons vu la structure
du fichier {\tt /etc/services}. Le fichier {\tt $/$etc$/$services} sert
{\`a} assurer cette correspondance entre num{\'e}ro de socket et nom de
service.
Il existe d'autres m{\'e}canismes d'annuaires sur les syst{\`e}mes {\Unix}
assurant cette fonctionnalit{\'e}~: le service NIS de Sun Microsystems.

Quelque soit le syst{\`e}me d'annuaire utilis{\'e}, on retrouvera, pour un num{\'e}ro
de r{\'e}seau donn{\'e}, les informations suivantes~:
\begin{itemize}
	\item	le nom officiel du service,
	\item	le num{\'e}ro de port,
	\item	le protocole de transport {\`a} utiliser,
	\item	les {\'e}ventuels noms secondaires.
\end{itemize}

{\bf Exemple~:}
\begin{quote}
\begin{verbatim}
ftp		21/tcp
telnet	23/tcp
http	80/tcp
\end{verbatim}
\end{quote}

%%%%%%%%%%%%%%%%%%%%%%%%%%%%%%%%%%%%%%%%%%%%%%%%%%%%%%%%%%%%%%%%%%%%%
\subsection{\label{nsyscall-sysc-svcs}Appels syst{\`e}mes et structures associ{\'e}s}

La biblioth�que de fonctions standards r{\'e}seau contient un certain nombre de
fonctions permettants d'obtenir les informations contenues dans le fichier
{\tt /etc/services} (ou {\'e}quivalent). On aura~:
\begin{description}
	\item[{\tt getservent()}]\mbox{}\\
		Les appels successifs {\`a} {\tt getservent()} permettent d'obtenir
		les entr{\'e}es disponible dans la table de traduction <<~nom de
		service $/$ num{\'e}ro de port $/$ protocole de transport~>>.
	\item[{\tt getservbyname()}]\mbox{}\\
		Permet d'effectuer une recherche et d'obtenir les informations
		de la premi{\`e}re entr{\'e}e en fonction d'un nom officiel pour un service donn{\'e}.
	\item[{\tt getservbynumber()}]\mbox{}\\
		A la m{\^e}me fonctionnalit{\'e} que la fonction {\tt getservbyname()}, mais
		la recherche s'effectue en fonction d'un num{\'e}ro de port.
\end{description}

La fonction {\tt getservent()} fonctionne de la m{\^e}me fa\c{c}on que les
fonctions {\tt gethostent()} et {\tt getnetent()}, c'est-{\`a}-dire qu'elle renverra
les informations sur l'entr{\'e}e courante de l'annuaire {\`a} chaque fois qu'elle
sera appel{\'e}e. Avant de rendre la main, elle pr{\'e}parera le contexte pour que le
prochain appel donne les informations sur l'entr{\'e}e suivante.Les fonctions permettant
de cr{\'e}er le contexte d'ex{\'e}cution sont~:
\begin{description}
	\item[{\tt setservent()}]\mbox{}\\
		Cette fonction initialise le contexte. Elle ouvre le fichier
		{\tt /etc/services} (ou tout autre syst{\`e}me d'annuaire) et positionne
		le pointeur de lecture au d{\'e}but. Le contexte est pr{\^e}t pour faire
		des appels successifs avec la fonction {\tt getservent()}.
	\item[{\tt endservent()}]\mbox{}\\
		Cette fonction ferme le contexte.
\end{description}

\begin{remarque}
De la m{\^e}me fa\c{c}on, {\tt getservbyname()} et {\tt getservbynumber()} n'influent pas
sur le contexte d'ex{\'e}cution positionn{\'e} par {\tt setservent()} et {\tt endservent()}.
Seule la fonction {\tt getservent()} l'utilisera.
\end{remarque}

Les fonctions {\tt getservent()}, {\tt getservbyname()} et {\tt getservbynumber()}
poss{\`e}dent les caract{\'e}ristiques suivantes~:
\begin{quote}
{\sl Synoptique~:}
\begin{quote}
\begin{verbatim}
pp =  getservent ();
pp =  getservnyname (name, proto);
pp =  getservbyaddr (port, proto);

struct servent *sp;
char           *name;
char           *proto;
int             port;
\end{verbatim}
\end{quote}

{\sl Avec~:}
\begin{quote}
\begin{tabular}{lp{8cm}}
{\tt sp}		&	Structure de type {\tt servent} d{\'e}crivant les informations sur
			le service correspondant {\`a} l'entr{\'e}e courante du fichier  {\tt /etc/services}.	\\
{\tt name}	&	Chaine de caract{\`e}res contenant le nom du service dont on d{\'e}sire obtenir les informations.\\
{\tt proto}	&	Chaine de caract{\`e}res contenant le nom du protocole de
			transport associ{\'e} au service dont on d{\'e}sire obtenir les
			informations.\\
{\tt port}	&	Entier contenant le num{\'e}ro du service dont on d{\'e}sire obtenir les informations.
\end{tabular}
\end{quote}

{\sl Codes de retour~:}
\begin{quote}
\begin{tabular}{lp{8cm}}
	Ok	& Renvoie un pointeur sur une structure de type {\tt servent}.	\\
	Erreur	& Renvoie le pointeur \verb=NULL=.\\
\end{tabular}
\end{quote}
\end{quote}

\begin{description}
	\item[{\tt getservent()}]\mbox{}\\
		Cette fonction ne poss{\`e}de pas d'arguments. Elle fonctionne en collaboration
		avec les fonctions {\tt setservent()} et {\tt endservent()}. Elle retourne un pointeur
		sur une structure de type {\tt servent}, structure
		contenant les informations sur une entr{\'e}e du fichier {\tt /etc/services}.
	\item[{\tt getservbyname()}]\mbox{}\\
		Cette fonction poss{\`e}de deux arguments~:
		\begin{itemize}
			\item le nom du service {\`a} rechercher,
			\item le nom du protocole de transport associ{\'e}.
		\end{itemize}
		Elle retourne un pointeur sur une structure de type {\tt servent}.
	\item[{\tt getservbynumber()}]\mbox{}\\
		Cette fonction poss{\`e}de deux arguments~:
		\begin{itemize}
			\item le num{\'e}ro du service {\`a} rechercher,
			\item le nom du protocole de transport associ{\'e}.
		\end{itemize}
		Elle retourne un pointeur sur une structure de type {\tt servent}.
\end{description}

La structure {\tt netent} est compos{\'e}e des champs suivants~:\\
\begin{tabular}{lp{8cm}}
	{\tt s\_name}		&	nom officiel du r{\'e}seau (chaine de caract{\`e}res).\\[1.5ex]
	{\tt s\_aliases}		&	un tableau de chaines de caract{\`e}res contenant la liste
					des noms secondaires du protocole. Le dernier {\'e}l{\'e}ment
					du tableau correspond au pointeur \verb=NULL=.\\[1.5ex]
	{\tt s\_port}		&	le num{\'e}ro du service.\\[1.5ex]
	{\tt s\_proto}		&	le nom du protocole de transport utilis{\'e}.\\
\end{tabular}

%%%%%%%%%%%%%%%%%%%%%%%%%%%%%%%%%%%%%%%%%%%%%%%%%%%%%%%%%%%%%%%%%%%%%
\section{\label{nsyscall-manipaddr}Manipulation des adresses IP}

%%%%%%%%%%%%%%%%%%%%%%%%%%%%%%%%%%%%%%%%%%%%%%%%%%%%%%%%%%%%%%%%%%%%%
\subsection{Conversions de notation}

Dans beaucoup de fonctions ou d'appels syst{\`e}mes, les adresses ou bien les
r{\'e}seaux IP doivent {\^e}tre repr{\'e}sent{\'e}s sous forme
num{\'e}rique. Par contre, la repr�sentation usuelle, pour plus de lisibilit{\'e}, est sous forme d'une chaine
de caract{\`e}res donnant la valeur d�cimale des quatre octets la composant.

De m{\^e}me, il est parfois n{\'e}cessaire de diviser l'adresse IP d'une machine en deux parties~:
\begin{itemize}
	\item	le num{\'e}ro de r{\'e}seau,
	\item	le num{\'e}ro de machine dans le r{\'e}seau.
\end{itemize}
On pourra aussi {\^e}tre amen{\'e} {\`a} effectuer l'op{\'e}ration inverse.

Il existe un ensemble de fonctions permettant d'effectuer les
conversions r{\'e}pondant {\`a} la majeure partie des besoins. Ces derni{\`e}res ne
s'appliqueront {\bf uniquement qu'aux adresses IP}. Elles utiliseront donc
la structure {\tt in\_addr} d{\'e}finie dans le fichier <<~{\tt netinet/in.h}~>>.

On trouvera~:\\
\begin{tabular}{lp{8cm}}

	{\tt inet\_addr(cp)}	&
	Convertit une adresse IP sous sa forme d{\'e}cimale, contenue dans une chaine
	de caract{\`e}res, en un entier long non sign{\'e} (repr{\'e}sentation num{\'e}rique de
	l'adresse IP) \\[1.5ex]

	{\tt inet\_ntoa(in)}	&
	Convertit une adresse IP sous sa forme num{\'e}rique (entier long non sign{\'e})
	en une chaine de caract{\`e}res la donnant sous sa forme d{\'e}cimale. \\[1.5ex]
	
	{\tt inet\_lnaof(in)}	&
	Renvoie la partie <<~num{\'e}ro de machine~>> {\`a} partir d'une adresse IP
	fournie sous sa forme num{\'e}rique (entier long non sign{\'e}) ou bien disponible
	via la structure {\tt in\_addr}.\\[1.5ex]
	
	{\tt inet\_netof(in)}	&
	Renvoie la partie <<~num{\'e}ro de r{\'e}seau~>> {\`a} partir d'une adresse IP
	fournie sous sa forme num{\'e}rique (entier long non sign{\'e}) ou bien disponible
	via la structure {\tt in\_addr}.\\[1.5ex]
	
	{\tt inet\_makeaddr(net,lna)} &
	Permet de construire une adresse IP sous forme de chaine de caract{\`e}res
	en notation d{\'e}cimale {\`a} partir d'un num{\'e}ro de r{\'e}seau et d'un num{\'e}ro de machine
	contenus dans deux variables num{\'e}riques.\\[1.5ex]
	
	{\tt inet\_network(cp)} 	&
	Convertit l'adresse d'un r�seau sous sa forme d{\'e}cimale contenu dans
	une chaine de caract{\`e}res, en un entier long non sign{\'e}. Cette adresse sera
	prise tel-quel. Au cas o{\`u} elle ne serait pas compl{\`e}te, seules les {\sl octets}
	sp{\'e}cifi{\'e}s seront convertis.\\
\end{tabular}

La structure {\tt in\_addr} correspond {\`a} un type {\tt union}, c'est-{\`a}-dire
que tous les champs sp{\'e}cifi{\'e}s dans sa d{\'e}finition recouvrent la m{\^e}me zone m{\'e}moire.
Le champ qui nous interressera le plus sera <<~{\tt s\_addr}~>>, correspondant {\`a} un 
entier long non sign{\'e} sur 32 bits. Il correspond donc {\`a} une adresse IP sous sa
forme num{\'e}rique. Pour plus de renseignements, reportez-vous au fichier
<<~{\tt /usr/include/netinet/in.h}~>> sur les syst{\`e}mes {\Unix}.

{\bf Exemples~:}
\begin{quote}
Les variables d{\'e}limit{\'e}es par <<~"~>> correspondent {\`a} des chaines de caract{\`e}res.
Les variables non d{\'e}limit{\'e}es correspondent {\`a} des variables num{\'e}riques.

Dans les exemples qui vont suivre, nous allons utiliser <<~{\tt input}~>> et
<<~{\tt output}~>> pour d{\'e}signer deux variables locales, dont le type d{\'e}pendra
de l'appel. Les colonnes <<~{\sl Entr{\'e}e}~>> et <<~{\sl Sortie}~>> indiqueront
leur contenu. Le contenu de la variable {\tt input} indiquera ce qui sera
pass{\'e} en argument {\`a} la fonction. Le contenu de la variable {\tt output} indiquera
le r{\'e}sultat obtenu apr{\`e}s appel.

La plupart des variables num{\'e}riques dans les exemples seront not{\'e}es en hexad{\'e}cimal
(chiffres pr{\'e}c{\'e}d{\'e}s de <<~0x~>>). On pourra ainsi voir la r{\'e}partition des nombres
au niveau des octets.

\begin{remarque}
Un entier long non sign{\'e} est cod{\'e} sur 32 bits. Par cons{\'e}quent, la notation d'un
tel nombre en hexad{\'e}cimal sera repr{\'e}sent{\'e}e par 8 chiffres.
\end{remarque}
\end{quote}

\begin{tabular}{|l|l|l|l|}
	\hline
	Fonction & Entr{\'e}e & Sortie & Exemple d'appel \\
	\hline \hline
	{\tt inet\_addr} 						&
		"127.0.0.6"					&
		0x7F000006					&
		{\tt output = inet\_addr(input)} 			\\
	\hline
	{\tt inet\_ntoa}						&
		0x7F000006					&
		"127.0.0.6"					&
		{\tt output = inet\_nota(input)} 			\\
								&
		2130706438					&
		"127.0.0.6"					&
								\\
	\hline
	{\tt inet\_lnaof}						&
		"192.254.1.5"					&
		0x00000005					&
		{\tt output = inet\_lnoaf(input)} 		\\
								&
		"126.1.10.6"					&
		0x00010A06					&
								\\
								&
		"138.21.17.2"					&
		0x00001F02					&
								\\
	\hline
	{\tt inet\_netof}						&
		"127.0.0.6"					&
		0x0000007F					&
		{\tt in.s\_addr = inet\_addr(input)} 		\\
								&
		"192.0.5.45"					&
		0x00C00005					&
		{\tt output = inet\_netof(in)} 			\\
								&
		"138.21.3.12"					&
		0x00008C15					&
								\\
	\hline
	{\tt inet\_makeaddr}					&
		127, 6						&
		"127.0.0.6"					&
		{\tt output = inet\_makeaddr(net, host)}	\\
								&
		35349, 274					&
		"138.21.17.2"					&
								\\
								&
		802821, 45					&
		"192.0.5.45"					&
								\\
	\hline
	{\tt inet\_network}					&
		"127.0.6"					&
		0x007F0006					&
		{\tt output = inet\_network(input)} 		\\
								&
		"138.21.17"					&
		0x008C1511					&
								\\
								&
		"192.0.5.45"					&
		0xC000052D					&
								\\
	\hline
\end{tabular}

%%%%%%%%%%%%%%%%%%%%%%%%%%%%%%%%%%%%%%%%%%%%%%%%%%%%%%%%%%%%%%%%%%%%%
\subsection{Conversion de type}

Dans le cas de sites h{\'e}t{\'e}rog{\`e}nes, il est possible que l'on puisse rencontrer quelques
difficult{\'e}s avec la fa\c{c}on dont le syst{\`e}me repr{\'e}sente les donn{\'e}es en m{\'e}moire.
Par exemple, certains syst�mes propri�taires inversent les octets 
de poid fort et de poid faibles. Sur d'autres syst{\`e}mes, aucune inversion n'est faite. La repr{\'e}sentation
d'entiers long peut donc changer.

Il est donc possible de rencontrer des probl{\`e}mes de protabilit{\'e}s d{\`e}s que l'on utilise
des donn{\'e}es sur 16 et 32 bits.

Pour {\'e}viter des probl{\`e}mes de protabilit{\'e}, il existe des {\sl macros} du langage C
assurant ce type de conversion. Dans le cas o{\`u} le syst{\`e}me proc{\`e}derait {\`a} une inversion,
celles-ci remettraient les donn{\'e}es dans le bon ordre. Dans le cas contraire, elles
n'effectueront aucune op{\'e}ration.

Les {\sl macros} disponibles sont~:\\
\begin{tabular}{|l|l|l|}
	\hline
		Macro & Entr{\'e}e & Sortie \\
	\hline
		{\tt htonl} & {\tt host long} & {\tt net long} \\
		{\tt htons} & {\tt host short} & {\tt net short} \\
		{\tt ntohl} & {\tt net long} & {\tt host long} \\
		{\tt ntohs} & {\tt net short} & {\tt host short} \\
	\hline
\end{tabular}